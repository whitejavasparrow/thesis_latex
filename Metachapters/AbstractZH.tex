\begin{document}

\chapter*{摘要}
\addcontentsline{toc}{chapter}{摘要}
本研究欲從量化/計算的觀點切入詞彙語意變遷的語言現象。近年來,文字在網路上大量流傳,加上社會快速變遷,語意表達亦不斷變化。與此同時,歷史文本的電子化亦開展了更多與歷時語意相關的研究可能,進而從中分析、挖掘詞彙所蘊含的詞意。 語言,將所思所想傳遞、紀錄,並在說話者使用語言時,不斷被重塑與流傳 \parencite[61]{blank1999new}。從詞意的改變、新舊字詞的興衰,探索其背後的運作機制與認知層面,進而得出語意變遷(semantic change)的規律性(regularities)\parencite[63]{blank1999new}。如果從共時(synchronic)的角度來看,語意存在各種變異(variation),而在歷時(diachronic)的脈絡下,經過時間累積而記錄著各式變遷。語料庫語言學以自然產生的語言使用資料為本,從中觀察、歸納出可質化、量化的語言分析,而歷時語料庫因應科技進步,計算語言學界亦已出現以詞向量(word embedding)、統計模型等方式探求語意在時間洪流下的變動與趨勢。 多義性(polysemy)是語意變遷另一大成因,詞彙將各個概念、意義的分類以語言的形式表達,語言共時下的詞義關係,時常亦已存於歷時的發展。漢語的詞彙組成從單字詞走向雙字詞(disyllabic words),不僅可以讓我們探究同音異義(homophony)的影響、字詞間的詞意互動、雙字詞與複合詞(compound)的增長。對於語料較稀少的歷史主題,計算語意學與統計模型的方法可突破許多困境,因為原始語料為寶貴研究材料,除此之外,有些詞彙雖然並無明顯的詞頻變化,其指涉對象與意義內涵卻與以往大不相同,在詞向量等詞彙表徵方法蓬勃發展之時,歷史語意變遷亦是不可缺少的研究主題。 在語料庫語言學的範疇,相關的研究主題被稱為「共現(co-occurrence)」方法,共現分佈的趨勢代表著意義分布的異同。以量化的方式量測語意變遷的程度,並以質化分析輔證已知的例子,並發掘更多可能的例子與規律。本研究以歷時語料庫(中國哲學書電子計畫 \parencite{sturgeon2019ctext})與現代漢語語料庫(中研院漢語平衡語料庫 \parencite{chen1996sinica})為語料來源,以歷時的詞向量搭配詞彙資料庫,了解單音節至複音節詞彙的語意變遷程度,\textcite{hamilton2016cultural} 的全域鄰近詞法,以搭配詞的相似度數值組成二階向量(second-order embedding),提高語意表徵的精確度,比較各時代向量的方法,其相關係數越低,語意變遷程度越高。此外,從詞彙的意義分布與互動,描繪出不同詞意的消長與變動。最後,本研究將採用以變異程度為基礎的近鄰群聚分析法(Variability-based Neighbor Clustering, VNC)\parencite{gries2012variability},此階層式的分群可勾勒出綜合性評估各觀察變項的影響下,漢語詞彙發展的時代區分。 計算語意學與歷史語意學的研究回溯驗證個別詞彙的意義變化,更進一步梳理整體的原理原則,詞彙反映人們對於新事物賦予新名、社會概念的更迭牽動詞彙之間的關聯。

\keywordsZH{語意變遷、歷時語意、向量表徵、階層式集群、資料驅動方法} % 5-7 keywords

\end{document}