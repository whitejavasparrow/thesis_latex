\begin{document}

\chapter*{Abstract}
\addcontentsline{toc}{chapter}{Abstract}
This research aims to investigate the topic of historical semantic change from the perspective of quantitative/computational linguistics. With a rapid accumulation of texts in the digital era, attention is called upon a more temporal-aware interpretation of language use and meaning construction. Meanwhile, the digitalization of historical texts opens up more research opportunities to trace the diachronic development of words and meanings. Especially, semantic change motivated by linguistic features and factors can be explored in a data-driven approach. Language is a means of communication through which ideas are conveyed, stored, and recorded, and in essence, constant change and evolution occurs as the speakers use the language with the passage of time \parencite[61]{blank1999new}. The dynamics of meaning construction is embodied in the emergence and losses of senses, as well as the split and shifts, which contributes to the different distributions and interactions of words, reflects the regularities and adaptability of the language, and the cognition and culture operating behind \parencite[63]{blank1999new}. Synchronic variations can be dealt with through a diachronic lens. Corpus-based, data-driven approach enables an observation and derived generalizations of semantic change. Coupled with the advances in vector space models and statistical analysis, the changes in meaning are explored. Polysemy is a driving force of semantic change. Concepts and meanings are structured in words and language use, and how word-formation is realized in Chinese is addressed in the development of monosyllabic to disyllabic words, which not only allows us to explore the influence of homophony, the interaction between words, and the growth of disyllabic words and compounds. Seeing that historical textual data are in demand, computational semantics and statistical models resolves the dilemmas. On top of that, it is possible that semantic change occurs not in observed frequency, but other distributional ways, making the encoded meanings distinctively different from previous time periods. As vector space models like word embeddings are receiving much attention, historical semantic change is a research topic that should enter the discussions. In the field of corpus linguistics, such research method are based on co-occurrences of words in context, and the co-occurrence distribution represents the similarities and differences in meaning interactions. The diachronic corpus consists of texts from the following sources: the Chinese Text Project \parencite{sturgeon2019ctext} and Academia Sinica Balanced Corpus of Modern Chinese for modern Chinese \parencite{chen1996sinica}. By applying a quantitative inquiry into semantic change, we will measure the degrees of semantic change, support known change cases, and discover unknown ones, with the consultation of lexical databases. Firstly, the global measures proposed by \textcite{hamilton2016cultural} is adopted. Second-order embeddings comprised of similarity scores of keywords are formed to compare the meaning representations of different eras. The lower the correlation between two temporally-adjacent vectors, the higher the degrees of semantic change. Secondly, based on the distribution and interaction of a word's senses, the semantic trajectories of the word will be traced. Finally, this study will proceed with periodization analysis using the Variability-based Neighbor Clustering (VNC) method \parencite{gries2012variability}. As a hierarchical clustering method, it is bottom-up, as opposite to the decisive clustering, a comprehensive evaluation of the influence of the selected linguistic factors in this study is implemented to explore how the development of meaning construction can be understood under different stages. In sum, this study explores the phenomenon of semantic change in retrospect to derive the semantic development in diachrony. The computational/statistical modeling of historical lexical semantic change will shed new light on how the language community describes and makes sense of the society that is also constantly changing.

\keywords{Semantic change, diachronic semantics, meaning representations, hierarchical clustering, data-driven approach} % 5-7 keywords

\chapter*{摘要}
\addcontentsline{toc}{chapter}{摘要}
本研究欲從量化/計算的觀點切入詞彙語意變遷的語言現象。近年來,文字在網路上大量流傳,加上社會快速變遷,語意表達亦不斷變化。與此同時,歷史文本的電子化亦開展了更多與歷時語意相關的研究可能,進而從中分析、挖掘詞彙所蘊含的詞意。 語言,將所思所想傳遞、紀錄,並在說話者使用語言時,不斷被重塑與流傳 \parencite[61]{blank1999new}。從詞意的改變、新舊字詞的興衰,探索其背後的運作機制與認知層面,進而得出語意變遷(semantic change)的規律性(regularities)\parencite[63]{blank1999new}。如果從共時(synchronic)的角度來看,語意存在各種變異(variation),而在歷時(diachronic)的脈絡下,經過時間累積而記錄著各式變遷。語料庫語言學以自然產生的語言使用資料為本,從中觀察、歸納出可質化、量化的語言分析,而歷時語料庫因應科技進步,計算語言學界亦已出現以詞向量(word embedding)、統計模型等方式探求語意在時間洪流下的變動與趨勢。 多義性(polysemy)是語意變遷另一大成因,詞彙將各個概念、意義的分類以語言的形式表達,語言共時下的詞義關係,時常亦已存於歷時的發展。漢語的詞彙組成從單字詞走向雙字詞(disyllabic words),不僅可以讓我們探究同音異義(homophony)的影響、字詞間的詞意互動、雙字詞與複合詞(compound)的增長。對於語料較稀少的歷史主題,計算語意學與統計模型的方法可突破許多困境,因為原始語料為寶貴研究材料,除此之外,有些詞彙雖然並無明顯的詞頻變化,其指涉對象與意義內涵卻與以往大不相同,在詞向量等詞彙表徵方法蓬勃發展之時,歷史語意變遷亦是不可缺少的研究主題。 在語料庫語言學的範疇,相關的研究主題被稱為「共現(co-occurrence)」方法,共現分佈的趨勢代表著意義分布的異同。以量化的方式量測語意變遷的程度,並以質化分析輔證已知的例子,並發掘更多可能的例子與規律。本研究以歷時語料庫(中國哲學書電子計畫 \parencite{sturgeon2019ctext})與現代漢語語料庫(中研院漢語平衡語料庫 \parencite{chen1996sinica})為語料來源,以歷時的詞向量搭配詞彙資料庫,了解單音節至複音節詞彙的語意變遷程度,\textcite{hamilton2016cultural} 的全域鄰近詞法,以搭配詞的相似度數值組成二階向量(second-order embedding),提高語意表徵的精確度,比較各時代向量的方法,其相關係數越低,語意變遷程度越高。此外,從詞彙的意義分布與互動,描繪出不同詞意的消長與變動。最後,本研究將採用以變異程度為基礎的近鄰群聚分析法(Variability-based Neighbor Clustering, VNC)\parencite{gries2012variability},此階層式的分群可勾勒出綜合性評估各觀察變項的影響下,漢語詞彙發展的時代區分。 計算語意學與歷史語意學的研究回溯驗證個別詞彙的意義變化,更進一步梳理整體的原理原則,詞彙反映人們對於新事物賦予新名、社會概念的更迭牽動詞彙之間的關聯。

\keywordsZH{語意變遷、歷時語意、向量表徵、階層式集群、資料驅動方法} % 5-7 keywords

\end{document}