\begin{document}

\chapter*{致謝辭}
感謝在撰寫論文的期間受到的各式幫助。

首先,謝謝我的指導老師謝舒凱老師,在我提出想要以「家」的概念演變為題目時,就一路指引我從各個角度切入這個主題,甚至深入到了千年歷史!在論文完成的過程中,對於我的疑惑不解總是無比包容,一步一步讓我接近知識的殿堂,同時也不忘像個點燈人,行程滿檔還常常關照我們,照亮可能有點晦暗的角落。

謝謝我的口試委員謝舒凱老師、呂佳蓉老師、以及張瑜芸老師,對論文提出許多珍貴的建議,從最初到成形,口委老師們的觀點和建議給予我很大的幫助和思考。

在語言所就讀的時光即將吿一段落了,感謝蘇老師、宋老師、江老師、呂老師、馮老師、李老師和邱老師,從老師們的教學中體會到語言學的領域是如此的廣闊無垠。謝謝美玲助教為所上大大小小的事務付出,在新冠肺炎期間尤其能感受到助教安穩的力量。

謝謝LOPE實驗室的大家,Taco, Hanna, Sabrina, Roxanne, Richard, Freya, Yolanda, Jessica, Ben, Don, Debby, 晴方, Yongfu, 飛揚, Derek, Amber, 貿昌, Simon, 智堯, 鈺琳, Joy, 昱翔學長。每個月的慶生和meeting常常會有的餅乾,輕鬆愉快的氣氛和滿足的肚子,在掙扎時總有許多援手幫忙

謝謝慧宇, 殷綮, Sherry, Ree, Sam, Brian, CJ,在進入碩班之後,在精進語言學這門領域的同時,看到很多很多多才多藝的同學,不論是統計、美術、咖啡等等,而每個學期結束的爬山行程也是面對paper和報告的一大動力;謝謝Thomas在歷史語言學及古漢語方面的慷慨相授,謝謝姿妤學姐在工作之餘還回到所上架所網,謝謝充滿音樂細胞的Andrew學長,一點一滴都豐富了我的碩班生活。

海格房的室友們 Irene, Wave, Eva, 乃甄, 小玉姐姐,收留這個想要有宿舍生活的我,北漂的你們很多時候比我還像台北人,帶我吃了好多地方。高中好友 Rachel, Sylvia, Sandy,忙碌的日子裡看到你們不斷地前進,儘管無法常常見面,也給予我很大的力量。Leston,煮好吃的食物共食,與我分享體壇新星,竟然都快比我們年輕了啊!謝謝Tom,在得知我要口試的時候,還送了我一件西裝外套,真的很感動!

我也想藉這個機會謝謝在大學時期讓我認識語言學的老師,薩文蕙老師一邊播著古典樂一邊做研究的模樣現在都還記得、鍾曉芳老師在某次課堂請我們喝綠豆沙,以及賴惠玲老師的提攜、Alyssa學姐對我的關心,都是使我成長的養分。

謝謝冠鳴,有什麼快樂喜悅都和我分享,有什麼需要幫忙的總是說好。很難得這段時間有你在旁邊讀著厚厚的書,我在鍵盤上敲敲打打,三隻貓不時窩在一旁,不時過來蹭蹭我,希望以後的日子我們也能繼續攜手一同前進。最後,謝謝爸媽與弟弟,在讀碩班全力支持我,讓我得以追求

\end{document}