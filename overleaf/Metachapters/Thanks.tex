\begin{document}

\chapter*{致謝辭}
因為論文進度緩慢,致謝辭也像是感謝日記一樣,一點一滴記錄下來,感謝在撰寫論文的期間受到的各式幫助。

首先,謝謝我的指導老師謝舒凱老師,在我提出想要以「家」的概念演變為題目時,就一路指引我從各個角度切入這個主題,在論文完成的過程中,對於我的任性總是無比包容,一步一步接近知識的殿堂,同時也不忘像個點燈人,照亮晦暗的角落。行程滿檔還常常關照我們,

謝謝我的口試委員謝舒凱老師、呂佳蓉老師以及張瑜芸老師,對論文提出許多珍貴的建議,看著論文從最初到成形,口委老師們的觀點和建議給予我很大的幫助和思考。

在語言所就讀的時光即將吿一段落了,感謝蘇老師、宋老師、江老師。謝謝美玲助教為所上大大小小的事務付出,在新冠肺炎期間尤其能感受到助教安穩的力量。

謝謝 LOPE 實驗室的大家,在掙扎時總有許多援手幫忙。Taco, Hanna, Sabrina, Roxanne, Richard, Freya, Ben, Don, Debby, Yolanda, Jessica, Yongfu, Andrea, 貿昌, Simon, Amber, 智堯, 鈺琳, 飛揚, Derek, Joy。每個月的慶生和meeting常常會有的餅乾,輕鬆愉快的氣氛和滿足的肚子,

慧宇, 殷綮, Sherry, Sam, Brian, Ree, CJ。每個學期結束的爬山行程就是學期的動力。

Alyssa, 海格房的室友們 Irene, Wave, Eva, 乃甄, 小玉姐姐,北漂的你們很多時候比我還像台北人,帶我吃了好多地方。高中好友 Rachel, Sylvia, Sandy,忙碌的日子裡看到你們不斷地前進,儘管無法常常見面,也給予我很大的力量。Leston,煮好吃的食物共食;謝謝Tom,在得知我要口試的時候,還送了我一件西裝外套,真的很感動!

Miffy, Thomas, 姿妤學姊,

% 謝謝在大學時期帶領我認識語言學的老師,薩文蕙老師、鍾曉芳老師、賴惠玲老師,對於

謝謝冠鳴,有什麼快樂都和我分享,有什麼需要幫忙的總是說好,很難得這段期間有你在旁邊讀著厚厚的書,我在鍵盤敲敲打打,三隻貓不時窩在一旁,不時過來蹭蹭我,希望以後的日子我們也能繼續攜手一同前進。最後,謝謝爸媽與弟弟還有家裡的三隻貓,在讀碩班全力支持我,

\end{document}