\begin{document}

\chapter{Introduction}
\section{Overview of the Computational-historical Inquiries into Semantic Change}
Language is constantly changing and evolving. The emergence of new senses, the demise of old ones, and the polysemous nature of linguistic expressions make the process of semantic change a dynamic phenomenon \parencite{robertinvanhove2008}. As individuals learn new words and meanings throughout their life, so does a language. As language users actively engage in processing and interpreting the language, the semantic history of words are woven into the texts that then survive time and are presented to us now. In the long run, a word is likely to convey a meaning completely different or unfathomable. For instance, ``the quick and the dead'', quoted from the Bible, means ``the living and the dead'', but the collective adjective ``the quick'' no longer makes sense in Present-Day English \parencite[199]{semanticincrowley2010}.

The nature of language is reflected in its use. In \citeyear{sinclair1982reflections}, \citeauthor{sinclair1982reflections} envisions the possibility of ``vast, slowing changing stores of text'' and ``detailed evidence of language evolution'' \ascitedin{renouf2002time}. In the recent years, a huge amount of historical text data have been digitized and made available to the public, and the use of digitized libraries as rich linguistic resources to observe how certain linguistic features are ``assimilated'' into the language becomes more and more feasible \parencite{renouf2002time}. While recent studies have used time-sliced collections of texts to observe swift meaning changes, the digitalization of texts from earlier time periods opens up research opportunities that incorporates a corpus-driven approach to trace the diachronic development of words and their meanings \parencite{kutuzov2018survey,tahmasebi2018survey,camacho2018survey}.

With the recent advances in Natural Language Processing (NLP) techniques, the changes in meaning over time can be to a great extent captured by representing discrete linguistic data as numeric vectors such as word embeddings, especially after the release of Word2vec \parencite{mikolov2013efficient}, GloVe \parencite{pennington2014glove} and FastText \parencite{bojanowski2016enriching}. For instance, for the study of semantic change of individual words across time, initial efforts have been put into generating word embeddings from different time spans and explore whether semantic change occurs based on the neighboring words of the target word from each time period.

As the pioneering computational-historical investigation in Mandarin, the monosyllabic word \zh{家}{jiā}{home} is selected as a case study in this thesis. The concept of home is an ancient, seemingly familiar and encompassing, but tangible one. Various humanities disciplines have sought to grasp the full picture. Defined by the Oxford English Dictionary (OED), the word \textit{home} is ``the place where a person or animal dwells'' \dictcite{homeinoed}. As one of the earliest 1\% entries to be included in the OED, this word has 35 main senses and 214 total senses—Home is a physical space, a place where we feel a ``sense of belonging [and] comfort'', and even a person's ``country or native land.'' In Mandarin Chinese, the MOE Revised Mandarin Chinese Dictionary defines its translated equivalent \zh{家}{jiā} as ``a place where family members live together (眷屬共同生活的場所)'', ``a private property (私有財產)'', and ``people in certain professional fields (經營某種行業或具有某種身份的人)'' \dictcite{jiainmoe}. Yet, how is the concept of home encoded linguistically? Specifically, how its diachrony interacts with synchrony and variations is the main concern of this study.

\section{Research Questions}
From the perspective of corpus-based computational linguistics, research questions are invoked as to how the concept of home is properly computationally represented? What words are semantically related to this concept? and how are these words co-construct the meanings of home, and how this concept comes into shape through the lens of time. In this study, the research questions are proposed as follows: (1) How can diachronic embeddings be applied to textual data of pre-modern Chinese? (2) Besides the linguistic factor of frequency change, how is semantic change of \jia be reflected in distributional ways? (3) What cultural implications can diachronic embeddings contribute to pre-modern Chinese? The research questions to be answered through a corpus-based case study approach along with diachronic word embeddings to investigate the evolution of meaning change in the target word \jia\rspace.

\section{Organization of the Thesis}
The remainder of this thesis is organized as follows. An theoretical overview and reflections of lexical semantic change in general, the concept of home in literature,as well as the diachronic word embeddings techniques are given in Chapter~\ref{related_works}. Chapter~\ref{methods} introduces the preprocessing issues, and the proposed corpus-based clustering method and distributed semantic representation models for the study. The development of word-level and sense-level word representations brings to the fine-grained analyses and generalizations of semantic change. Chapter~\ref{results} describe how the proposed approaches are evaluated, and showcase analyses made possible by our approach, and discusses their successes and limitations. Finally, Chapter~\ref{conclusions} concludes with a summary of the contributions and with considerations on the future works as well as on its usefulness to linguistic investigations and other social-cultural  applications.

\end{document}