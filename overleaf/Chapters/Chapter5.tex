\begin{document}

\chapter{Conclusions}
\label{conclusions}

In light of the growing research interest in diachronic semantic change, this thesis is a case-study investigation of the keyword \jia through a corpus-based approach. The evolution of \jia is a compressed history of the Chinese society and the Chinese language. As linguistic change may not necessarily be reflected in abrupt frequency change when the time scope is a long stretch, a historical-computational analysis into the research topic can provide further insights into the phenomenon of semantic change by experimenting diachronic semantic modeling and taking into consideration how semantic change interacts with various linguistic factors.

To capture semantic change that might not be accompanied by change in frequency, but in other distributional ways, the analysis of diachronic embeddings on the word-level of \jia serves as a starting point to pinpoint the core, stable meanings of the word, outlining the properties of a physical space and a structured social unit. While an observation is made about the economic situation from pre-modern time, the word \jia becomes less associated with individuated roles such as a wife, but emphasizes more on the self, depicting personal memories and living experiences about home, as well as the transition of family types from family lineage to a smaller household unit. In the field of corpus and computational linguistics, changes of word choice and the inclusion of more senses or vice versa allow for a closer look at the texts in snapshots of specific time frames, while resonate with studies in other disciplines.

With the advantage of distributional semantic models, the interchangeable use of the words \textit{home}, \textit{house}, and \textit{family} can be explored as different components under an umbrella concept. Especially, pre-modern Chinese is distinguished from the current written form, with a vocabulary consisting of different lexical items, and disyllabic words occupying less proportion of the whole vocabulary in pre-modern Chinese than in modern Chinese. The disparity results in the addition and (dis)appearance of senses of the single-character word \jia\rspace , and aspects of meanings are encoded in different two-character words in the modern time.

Additionally, the analysis of diachronic embeddings on the sense-level depicts the interaction of senses in the form of competition and cooperation. However, as discussed in \textcite{giulianelli2019lexical}, one of the challenges to apply pre-trained language models to diachronic or historical textual data is the fine-tuning of these large-scaled models, for the models might fail to yield satisfactory results of temporal-specific contextualized usage/token representations. As hinted by \textcite{giulianelli2019lexical}, the fine-tuning is performed based on a classification task of recognizing the time period of a portion of documents, but the fine-tuned models might instead reflect the style of prominent authors of certain time periods, veering away from baseline representations. Faced with these problems, \textcite{kutuzov2020uio} also compares contextualized embeddings with context-independent ones, and find that for semantic change detection, context-independent embeddings are as effective overall.

Indeed, the modeling of semantic change is an approximation and simulation of the linguistic phenomenon by examining the relationships between various linguistic factors. It is even more pioneering to delve into the interface between phonology and semantics. How polysemy of homophones is to be explored through external resources such as dictionaries and negative examples \textcite[15]{traugott2001regularity} is a question that remains unanswered.

Language does not cease to change beyond the observable texts within the time frame of the chosen corpora. As language use is a dynamic process and phenomenon, semantic change modeling has profound impacts in linguistic analysis irregardless of whether the time is fast-forward or is perceived in retrospect, and a temporal-aware understanding of diachronic semantic modeling is a starting point of research. Following the examination of linguistic factors, sense evolution prediction, the relations between semantic change and different linguistic, cultural factors may deepen our understanding of word composition and meaning construction. The task of meaning representation from the perspective of semantic change is especially rewarding toward how the modeling of meaning representation can be tweaked, unlike in English, in which it is not always the case that preprocessing of compound words are taken into account from the beginning. Therefore, it is hoped that this study could offer a cross-linguistic and metalinguistic analyses of historical-computational semantic modeling.

In this study, word-level and sense-level embeddings serve as a starting point to investigate the semantic development of Chinese, which is so distinctively different in pre-modern and modern time that a call for a special focus on the disyllabic development of Chinese is needed to account for the differences in different time periods. Recently, a number of dependency parsers for pre-modern Chinese have been released, yet it is found that the segmentation has to rely on textual data of a specific range of time. Nonetheless, through the analysis of different measures of semantic change, this study captures different aspects of semantic properties, and it is hoped that the results can lay an empirical basis of how single characters behave semantically by considering the time dimension of the textual data, especially the aspects of polysemy and multi-word expressions. In conclusion, this study aims to explore the meaning representations that are more dynamic than present application is populated for, and to show how word co-occurrences can be revealing in terms of such a concept like home that is relatively stable but ever-evolving with the passage of time.

The importance of temporal-aware, diachronic embeddings have been stressed both for modern texts and historical ones \parencite{huang2019neural,rosin2017learning,ruder2017word}. With the accumulation of texts in corpora, diachronic embeddings are more and more commonly used, i.e., for a search system to answer ``when'' two terms are related to each other, query expansion, and weighted synonyms \parencite{rosin2017learning}. As also already pointed out in the application of \acrlong{tot} (\acrshort{tot}), temporal-aware meaning representations are beneficial to reading comprehension and background settings, as well as event extraction that exhibit the dynamics of entities involved \parencite{wijaya2011understanding}. It is by this aim that this study is motivated, and for the purpose of achieving more understanding of the properties of language use through the lens of time. Furthermore, the rate of semantic change is another important issue so as to incorporate ``time-sensitive'' query expansion \parencite{rosin2017learning} to involve the time dimension of the linguistic phenomenon more in this rising, flourishing field of study.

As researches combine textual data from various corpora or linguistic sources, the detection of semantic change and measurement of degrees of change help satisfy time-specific needs, which have become increasingly fundamental and critical with an abundance of textual data presented to us nowadays. The analysis can be further explored by reaching out to other research disciplines and communities, and even the design and functionality of a diachronic corpus itself.
\end{document}