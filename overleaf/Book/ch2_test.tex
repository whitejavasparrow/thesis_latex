
\section細說構詞的故事]{\rmfamily 第二章 形態學[FF1A?]細說構詞的故事}
\rmfamily
連金發

\rmfamily
國立清華大學語言學研究所

\begin{listWWviiiNumxleveli}
\item \begin{styleqwerty}\rmfamily
前言
\end{styleqwerty}
\item \begin{styleqwerty}\rmfamily
語詞和詞組
\end{styleqwerty}
\item \begin{styleqwerty}\rmfamily
音位與形態音位
\end{styleqwerty}
\item \begin{styleqwerty}\rmfamily
形態和漢語類型演變
\end{styleqwerty}
\item \begin{styleqwerty}\rmfamily
形態和語言系統中其他部門的互動
\end{styleqwerty}
\item \begin{styleqwerty}\rmfamily
複合和派生的區別
\end{styleqwerty}
\item \begin{styleqwerty}\rmfamily
從複合到派生
\end{styleqwerty}
\item \begin{styleqwerty}\rmfamily
形態和語言層次的互動
\end{styleqwerty}
\item \begin{styleqwerty}\rmfamily
‘个’的派生功能
\end{styleqwerty}
\item \begin{styleqwerty}\rmfamily
重疊式
\end{styleqwerty}
\item \begin{styleqwerty}\rmfamily
台灣南島語的形態
\end{styleqwerty}
\item \begin{styleqwerty}\rmfamily
摘要與結論
\end{styleqwerty}
\item \begin{styleqwerty}\rmfamily
參考文獻
\end{styleqwerty}
\end{listWWviiiNumxleveli}

\rmfamily
形態與語言系統中的其他部門有密切的互動關係

\subsection{\rmfamily 2.1 前言}

\textrm{本章旨在介紹「形態學(又稱詞法)」(morphology)的基本概念和解析法}。\textrm{語言素材主要是台灣閩南語(簡稱台語)}[FF0C?]\textrm{並拿中文做對照[FF0C?]以彰顯兩者的異同}。\textrm{主要內容參閱文章開頭的要目}。\textrm{此外}[FF0C?]\textrm{台灣南島語的形態有其獨特引人之處}[FF0C?]\textrm{也略加著墨。}

\textrm{「形態學」是語言學歷史最悠久的研究領域之一。古希臘、拉丁語和梵文都是富於屈折變化的語言[FF0C?]語詞形態變化的問題向來是研究的焦點。形態學(或稱詞法)一向是語言體系中不可或缺的一環[FF0C?]與句法、音韻、詞彙、語意、語用產生重要的互動關係。本文架構如下: 2.2節指出語詞和詞組的分野[FF0C?]2.3節討論語詞的組成單位「音位」及「形態單位」[FF0C?]2.4節從形態的觀點探討漢語類型的演變[FF0C?]2.5節探究形態和語言系統中其他部門的互動[FF0C?]2.6節區分複合和派生[FF0C?]2.7節討論複合到派生的演變[FF0C?]2.8節探索形態和語言層次的互動[FF0C?]2.9節說明「个」的派生功能[FF0C?]2.10節論各種重疊式[FF0C?]2.11節討論台灣南島語的各種形態功能[FF0C?]2.12節做總結。}

\subsection{\rmfamily 2.2 語詞和詞組}

\textrm{Bloomfield認為「形態學」研究詞的內部結構[FF0C?]而「句法」(syntax)研究詞和詞的組合關係。因此「語詞」(word)和「詞組」(phrase)的不同反映形態學和句法的分野。形態學研究的對象涵蓋「屈折變化」(inflection)(又稱「活用」)、「派生」(derivation)、「複合」(compounding)、「重疊」(reduplication)。換言之[FF0C?]屈折變化是詞位為了表示句法、語義範疇之聚合關係而產生的「形態」(morphological)變化[FF0C?]反映形態學和音韻、句法、語義的互動[FF0C?]派生是語根加綴的一種形態作業方式[FF0C?]複合是「實詞」(lexeme)相加的一種形態組合[FF0C?]重疊是依語根的基模複製形態的方式[FF0C?]形式隨基模而變。請注意lexeme理解為「實詞」是相對於「虛詞」而言(詳2.4.節)[FF0C?]但理解為「詞位」是指詞彙(lexicon)的基本單位[FF0C?]如「書}\textrm{\textsubscript{1}}\textrm{」和「書}\textrm{\textsubscript{2}}\textrm{包」的「書}\textrm{\textsubscript{1}}\textrm{」(讀用)和「書}\textrm{\textsubscript{2}}\textrm{」(連用)的聲調不同[FF0C?]即前者高平調[FF0C?]後者中平調[FF0C?]但是都歸為同一个詞位。其實既使兩者因位置的變動而調形不同[FF0C?]兩者的調位(toneme)都是陰平調。此外[FF0C?]一個形態也可以從不同的角度來看待[FF0C?]如「黃」 ng}\textrm{\textsuperscript{5}} \textrm{(國際音標為/ŋ̩/) (如暫不考慮聲調)既是一個音節(syllable)[FF0C?]也是一個詞位(lexeme)}。\textrm{音節是指音韻結構}[FF0C?]\textrm{詞位是指詞彙中最基本的音義合體單位}。\textrm{/ŋ̩/ 甚至可以看成是一個音位(phoneme)}[FF0C?]\textrm{即音韻系統中不具語意}[FF0C?]\textrm{但有辨義作用(distinctive)的最基本的單位}。\textrm{一個形態[FF0C?]如「馬」}[FF0C?]\textrm{又是形位又是詞位或實詞}。\textrm{「馬」這個例子都合乎三個術語的定義}。\textrm{就是說}[FF0C?]\textrm{同一個對象可從不同的觀點去理解}。\textrm{屈折變化是反映句法範疇的形態變化}[FF0C?]\textrm{而派生(derivation)指稱語詞內部的形態變化}。\textrm{現代中文不具屈折變化}[FF0C?]\textrm{等到論及台灣南島語我們再來討論}。

\subsection{\rmfamily 2.3 音位與形態音位}

\textrm{語言是一種「形式」(form)}[FF0C?]\textrm{而非純粹是「實質」(substance)}。\textrm{同一個素材在不同的體系或層次中扮演不同的腳色。音韻和形態分屬不同的層次:比如閩南語‘姑’ koo}\textrm{\textsuperscript{1}}\textrm{和‘古’ koo}\textrm{\textsuperscript{2}}\textrm{是一個「最小的配對」(minimal pair)}[FF0C?]\textrm{兩者唯一的差別是聲調的不同[FF0C?]一個是高平調[FF0C?]另一個是高降調。上加符號是代表古代的聲調[FF0C?]1表示陰平調[FF0C?]2表示陰上調。聲調作為一種「音位」(phoneme)}[FF0C?]\textrm{不具語義[FF0C?]卻是能區辨詞匯意義的「區別」(distinctive)單位。(本文的音標根據教育部台灣閩南語辭典)它是音韻系統中最基本的單位, 這裡高平調和高降調的替換可以用來區辨詞匯的不同[FF0C?]但是高平調和高降調也可以用來表現形態音位的變體[FF0C?]如‘好’ ho}\textrm{\textsuperscript{2}}\textrm{和‘好看’ hoo}\textrm{\textsuperscript{2}}\textrm{khuann}\textrm{\textsuperscript{3}}。\textrm{‘好’的聲調有兩種唸法。‘好’ ho}\textrm{\textsuperscript{2}}\textrm{單獨讀為51}[FF0C?]\textrm{與後字連用時讀做55}。\textrm{一個詞兩個讀音[FF0C?]反映出該詞是一個抽象的單位[FF0C?]體現為兩個互補的聲調形態[FF0C?]稱為「形態變體」或「形態音位」(morphophoneme)}。\textrm{總之‘姑’ koo}\textrm{\textsuperscript{1}}\textrm{和‘古’ koo}\textrm{\textsuperscript{2}}\textrm{是兩個不同的詞[FF0C?]靠調位來區別[FF0C?]而‘好’的55和33是同一個「形位」(morpheme)的兩個不同的調形[FF0C?]彼此互補。}

\subsection{\rmfamily 2.4 形態和漢語類型演變}

\textrm{上古漢語或遠古漢語時期有些語法現象是由語根內部的形式變化表示的[FF0C?]其中包括音段(元音、輔音)和超音段(聲調)的變化。比如[FF0C?]兩個語詞可以藉聲母的清濁分出自動他動[FF0C?]聲調的交替也可以表示不同的語法關係[FF0C?]比如台語還殘存古清濁聲母交替反映致使非致使的對立[FF0C?]如「斷臍」 tng}\textrm{\textsuperscript{2}} \textrm{tsai}\textrm{\textsuperscript{5}} \textrm{(使臍斷)對「斷去」tng}\textrm{\textsuperscript{7}} \textrm{khi}\textrm{\textsuperscript{0}}\textrm{(斷掉)}[FF0C?]\textrm{即前者陰上的「斷」帶古清聲母[FF0C?]後者陽去的「斷」帶古全濁聲母。去聲和入聲的交替除了聲調的差異還涉及韻尾的不同。}

\begin{stylei}
根據「內部擬構法」(internal reconstruction)我們可以假定更早的階段[FF0C?]語根形式原來一致[FF0C?]清濁交替的產生是由於鼻音前綴使其中一個語根的聲母濁化之結果[FF0C?]去聲和入聲的交替也可能是加綴的結果[FF1B?]去聲可能是由入聲加後綴而來的[FF0C?]即原來入聲的塞音韻尾因後綴的影響而失落。
\end{stylei}

\textrm{就語言系統中形式和語法範疇之對應關係而言上古漢語音段或超音段可以充當音位或形位[FF0C?]充當形位時表示兩個詞之間的形態關係。但是現代漢語音段或超音段除了少數不具孳生力的例外都喪失了形態的功能[FF0C?]由音節擔負起替代的角色。約言之[FF0C?]音節透過「派生」(derivation)}、\textrm{「複合」(compounding)}、\textrm{「重疊」(reduplication)的方式起形態的作用。上古時期音節的內部結構較複雜[FF0C?]單音節單用時較不會因「同音」(homophony)而產生溝通困難的現象[FF0C?]可是一旦音節結構因音變(如複輔音變單輔音[FF0C?]聲母清濁對立消失)趨於簡化[FF0C?]單音節的同音機率就相對的增高[FF0C?]語言系統自有一套適應機制[FF0C?]不論是「派生」、「複合」或者「重疊」都是造就多音節詞的手段。「派生」是語根「加綴」(affixatioin)的造詞方式[FF0C?]中文「虎」不能單說[FF0C?]非要加上前綴「老」不可[FF0C?]但台語「虎」可以單說[FF0C?]究其原因[FF0C?]可能是台語的聲調較多[FF0C?]共七個調類[FF0C?]有助於語詞的區辨之故。另外同意並列複合詞的存在也見證了現代漢語(包括中文和台語)趨向多音節化[FF0C?]比如「朋友」這個複合詞是由兩個語意相近的「實詞」(lexeme)組成的。如果只就實詞是音義的合體而言[FF0C?]其中一個成員就足夠了}[FF0C?]\textrm{犯不著由兩個音節表示。形位是語言中形態部門能起造詞作用的最小單位[FF0C?]本身不必然具備詞彙意義[FF0C?]如中文「老虎」的「老」是不帶詞彙意義的形位[FF0C?]或更精確地說為「純形位」(morphome) (詳O’\citealt{Neill2013})}[FF0C?]\textrm{這裡的「老」不是年紀大的意思[FF0C?]比較「小老虎」[FF0C?]即「老」的作用充其量只是使詞根「虎」和前綴「老」結合成可「獨用的」(free)「語詞」(word)}。\textrm{「詞位」(word)是詞彙中可以獨用的基本單位}[FF0C?]\textrm{可以是實詞(lexeme)或虛詞(或稱功能詞)}。\textrm{兩者的區分可以從兩方面來說}。\textrm{一來}[FF0C?]\textrm{實詞是形義兼備的實體}[FF0C?]\textrm{虛詞只具語法功能}[FF0C?]\textrm{不帶詞彙意義}[FF0C?]\textrm{如只具造詞功能}[FF0C?]\textrm{就是純形位}。\textrm{二來}[FF0C?]\textrm{實詞是語言中不能窮舉的(inexhaustible)詞彙}[FF0C?]\textrm{屬開放類, 如名詞、動詞[FF0C?]虛詞是語言中數量有限的封閉類}[FF0C?]\textrm{如指代詞}、\textrm{體貌標記}、\textrm{語助詞(particle)(包含句末助詞)等}。\textrm{形位作為形態部門的最基本的單位可以是語根或詞綴}。\textrm{句末助詞雖然是連用成分}[FF0C?]\textrm{但相結合的對象是語句或詞組而非語根}。\textrm{「語詞」(word)是一個籠統通俗的稱號[FF0C?]可以指詞位、功能詞或音韻詞 (phonological word)}。\textrm{音韻詞是指語詞的音韻結構特性[FF0C?]如台語「一」it}\textrm{\textsuperscript{4}}\textrm{、「六」liok}\textrm{\textsuperscript{8}}\textrm{、「七」tshit}\textrm{\textsuperscript{4}}\textrm{、「八」pat}\textrm{\textsuperscript{4}}\textrm{、「十」sip}\textrm{\textsuperscript{8}}\textrm{(文讀)這五個數詞都是入聲語詞[FF0C?]即帶清塞音韻尾的音節。}

\textrm{大體而言[FF0C?]上古時期音段或超音段又是音位又是形位[FF0C?]單音節單位多半是「詞位」(word)}[FF0C?]\textrm{多音節(聯綿詞或外來語除外)單位是「詞組」(phrase)}[FF1B?]\textrm{現代漢語音段或超音段是音位[FF0C?]單音節是形位[FF0C?]多音節是詞位也可能是詞組。從語言類型的發展來說[FF0C?]上古漢語到現代漢語是從「綜合」(synthetic)語變為「分析」(analytic)語[FF0C?]綜合語語詞形義具不對稱的關係[FF0C?]分析語形義之間較具一對一的對稱關係。}

\subsection{\rmfamily 2.5 形態和語言系統中其他部門的互動}

\textrm{語言系統由「音韻」(phonology)}、\textrm{「形態」(morphology)}、\textrm{「詞匯」(lexicon)}、\textrm{「語義」(semantics)}、\textrm{「句法」(syntax)等各個自主而彼此有連帶關係的部門組合而成[FF0C?]每個部門都有一個最基本的單位[FF0C?]「音位」(phoneme)}、\textrm{「形位」(morpheme)}、\textrm{「實詞」(lexeme)}、\textrm{「義位」(sememe)}、\textrm{「法位」(taxeme)分別構成音韻}、\textrm{形態}、\textrm{詞匯、語義、句法部門最基本的單位。}

\subsubsection{\rmfamily 2.5.1 形態和詞匯的互動}

\textrm{「詞匯」(lexicon)是語詞的匯集[FF0C?]通常認為是不規則語言單位的集合。在人的大腦中必須由記憶部門處理[FF0C?]但是形態學卻是由規律運作。詞匯可以分成「現存」(real)詞和「可能」(potential)詞[FF0C?]可能詞是由形態學的規律產生的[FF0C?]不必記憶。詞匯區分規則和記憶部門可以解釋造詞中的創新性。形態學造出來的可能詞必須拿來和現存詞對照[FF0C?]根據「卡位」(blocking)原則[FF0C?]造出的可能詞如果和現存詞語義完全一致[FF0C?]就會被排擠掉[FF0C?]如‘台灣大學’一旦簡稱‘台大’}[FF0C?]\textrm{台北大學就不能再簡稱‘台大’}[FF0C?]\textrm{而需另謀出路[FF0C?]簡稱‘北大’}。\textrm{排擠效應也可能起因於同音[FF0C?]如有‘系友會’而很難有‘所友會’}[FF0C?]\textrm{因為已經有現存詞‘所有’}。\textrm{即便使用 ‘所友會’}[FF0C?]\textrm{還是會有揮不去‘所有’這個同音詞干擾的影子。此外[FF0C?]中文裡准後綴‘度’附著於形容詞後面可以造出像‘高度’}、\textrm{‘長度’}、\textrm{‘寬度’}、\textrm{‘厚度’}、\textrm{‘深度’}、\textrm{‘密度’}、\textrm{‘強度’}、\textrm{‘濃度’}、\textrm{‘硬度’等名詞[FF0C?]但是卻沒有*‘大度’這樣的名詞、這可能是詞匯中已經有現存詞‘尺寸’‘大小’之故。排擠效應並非絕對的[FF1B?]對現存詞的熟悉度越高排擠效力就越強。從現存詞和可能詞的競爭關係可以看出形態和詞匯的各種互動現象。上述規則和記憶的關係尚需補充說明。英文規則的複數形/-s/, /-z/, /iz/固然只需記住其出現的音韻條件即可[FF0C?]因為那些語詞是開放類[FF0C?]不能窮舉[FF0C?]但是這三個複數形態還是需要記憶的。此外[FF0C?]單複數常帶不同的語義[FF0C?]如 leaf 有不規則複數形 leaves 和規則複數形 leafs}[FF0C?]\textrm{特別是規則複數形可能跟隱喻或抽象的語義有關[FF0C?]如 Maple Leafs (楓葉隊)。綜上所述[FF0C?]也許除了實詞外我們還需要另一個概念:「錄位」(listeme) 即記憶的單位[FF0C?]錄位不拘大小[FF0C?]可以是語段[FF0C?]句子、詞組、單詞、音段、元音、輔音、聲調等。}

\subsubsection{\rmfamily 2.5.2 形態和句法的互動}

\textrm{有的語言學家認為複合詞和派生詞的不同在於前者有「詞內部的組合」(word-internal syntax)關係[FF0C?]而後者沒有[FF0C?]即派生詞是由語根加綴形成的。有的語言學家卻主張[FF0C?]不論是複合詞和派生詞[FF0C?]其內部的結構都與句法有淵源的關係。形態學是句法的投射。這一點可以從「論元結構」(argument structure)的承續和變化看出來[FF0C?]論元結構是「概念結構」(conceptual structure)和「句法結構」的介面[FF0C?]比如‘可愛’}、\textrm{‘可憐’}、\textrm{‘可恨’}、\textrm{‘可疑’}、\textrm{‘可笑’}、\textrm{‘可取’等「可+V」准派生詞其中的V是及物動詞[FF0C?]它的句法來源應該是「主語+V+賓語」的格式[FF0C?]主語充當「外元」(external argument)}[FF0C?]\textrm{賓語充當「內元」(internal argument)}。\textrm{及物動詞前頭加‘可’}[FF0C?]\textrm{使內元「外化」(externalize)變成外元[FF0C?]原來的外元隱藏起來。如‘旺旺真可憐’中的主語‘旺旺’是原來充當感受對象的內元外化而來的[FF0C?]原來充當感受者的外元隱而不顯。這樣的變化很像「中間語態」(middles)}[FF0C?]\textrm{這種准派生詞都可看成帶有反映主語屬性的形容詞[FF0C?]前面可以用副詞修飾就是明證。准前綴‘可’可以視為詞性是形容詞的「中心語」(head)}[FF0C?]\textrm{決定整個派生詞的詞類。文言文中‘可’單用是「情態詞」(modal)}[FF0C?]\textrm{但是在上述的形容詞中‘可’雖是前綴[FF0C?]卻是形容詞性的中心語[FF0C?]和動詞性的語根結合時[FF0C?]產生形容詞[FF0C?]這點和英文的後綴–able 相似[FF0C?]able 具形容詞的屬性且為中心語[FF0C?]和動詞eat結合[FF0C?]形成形容詞eatable}。\textrm{句法絕大多數都是「同心的」(endocentric)}[FF0C?]\textrm{即中心語能決定整個組合的詞性[FF0C?]但詞法中的複合詞有些有中心語[FF0C?]有些沒有。「非同心」(或稱異心exocentric)的情況正可用來作為區辨複合詞和詞組的依據。比如台語的‘炒米粉’結構上有兩解: 詞組[FF0C?]如‘在炒米粉’ (同心結構)和複合詞[FF0C?]如‘一盤炒米粉’(異心結構)}。\textrm{台語的‘米粉炒’只能有複合詞的解讀。}

\subsubsection{\rmfamily 2.5.3 形態和語用的互動}

\textrm{有些語詞表達「認知」(cognitive)意義[FF0C?]有些語詞只帶「情緒」(affective)意義[FF0C?]有些語詞既沒有認知意義也沒有情緒意義[FF0C?]即所謂有形無義的「空形位」(empty morpheme)[FF0C?]但是有時同一個形位三種用法兼而有之[FF0C?]中文裡‘老’就是一個例子。‘老鼠’的‘老’不帶任何意義只是充當造詞素材的空形位。空形位本身不帶詞彙意義[FF0C?]卻能使不能獨用的‘鼠’成為語詞。這裡前綴‘老’是形位[FF0C?]不是「語詞」(word)。語詞是形義兼備的合體且能獨用。現代中文‘鼠’形義兼備[FF0C?]但不能獨用[FF0C?]不能成詞。古文‘鼠’可獨用[FF0C?]自然成詞。‘老’有多重的用法[FF0C?]如當頻率副詞[FF0C?]如‘老罵人’}[FF0C?]\textrm{當形容詞[FF0C?]如‘菜很老’}[FF0C?]\textrm{這兩種用法都算是名符其實形義兼備的語詞[FF0C?]但有些情況[FF0C?]‘老’雖是形位[FF0C?]但不是空形位[FF0C?]而是帶有其他功能。‘老朋友’的‘老’是形容詞[FF0C?]但其性質很像副詞[FF0C?]‘老朋友’是指具有持久的友誼關係的人[FF0C?]而非指年紀大的朋友[FF0C?]另一種用法的‘老’帶情緒意義[FF0C?]如‘老王’}。\textrm{帶情緒意義的語詞用來表示「交談者」(interlocutor)(即「發話者」(speaker)和「受話者」(addressee))之間的愛憎、親疏、尊卑的關係[FF0C?]表現出某種「語用的」(pragmatic)信息。使用‘老王’這個稱謂語反映出說話者的社會地位比對方還高或相等[FF0C?]且兩者具有親近的關係。‘老王’是前綴加語根的結構[FF0C?]由此可以看出[FF0C?]加綴作為一種構詞手段也和語用信息有互動的關係。這種情緒的用法已經成為「成規義涵」(conventional implicature)[FF0C?]不能算是「會話義涵」(conversational implicature)。}

\textrm{複合詞內部語義關係相當多樣。論者有的想提出一套語義類型來概括構件之間所有的語義關係[FF0C?]但是語言中具有無限的「創新」(creativity)的可能[FF0C?]複合詞內部語義關係變化萬千[FF0C?]隨著這語言外的「情境」(context)而變[FF0C?]不能被固定的語義關係所束縛[FF0C?]比如pumpkin bus的意義可依語境而變[FF0C?]或指外表像南瓜的公車、或指載運南瓜的公車[FF0C?]甚至多一分想像空間可指載乘客去某站購買南瓜的公車。}

\subsection{\rmfamily 2.6 複合和派生的區別}

\textrm{「複合」(compounding)如界定為「獨用形位」(free morpheme)的組合[FF0C?]可能不夠周全。與其如此定義不如說是「實詞」(lexeme)的組合較為穩妥。「派生」(derivation)是「語根」(root)加上「詞綴」(affixation)的造詞方式。語根必然是實詞[FF0C?]詞綴(包括前、中、後綴)必然是虛詞。如表一所示[FF0C?]「複合」和「派生」不能用「獨用」(free)和「連用」(bound)的檢驗標準來區分。}

\textrm{表2-1 複合、}派生對照表

\tablefirsthead{}

\tabletail{}
\tablelasttail{}
\begin{tabularx}{\textwidth}{XXXXXX} & 複合 & 例子 &  & 派生 & 例子\\
\lsptoprule
& 實詞+實詞 &  &  & 實詞+虛詞 & \\
 1a & 獨用+獨用 & {\sffamily \textrm{風吹hong}\textrm{\textsuperscript{1}}\textrm{tshe}\textrm{\textsuperscript{1}}\textrm{(風箏)}}

{\sffamily \textrm{、銃子 tshing}\textrm{\textsuperscript{3}}\textrm{tsi}\textrm{\textsuperscript{2} }\textrm{(子彈)}} & 2a & 獨用+連用 & {\sffamily \textrm{穡頭sit}\textrm{\textsuperscript{4}}\textrm{thau}\textrm{\textsuperscript{5}}\textrm{(工作)、歲頭he}\textrm{\textsuperscript{3}}\textrm{thau}\textrm{\textsuperscript{5}} \textrm{(歲數)}}\\
 1b & 獨用+連用 & {\sffamily \textrm{囝兒 kiann}\textrm{\textsuperscript{2}}\textrm{ji}\textrm{\textsuperscript{5}} \textrm{(子女)、新婦sim}\textrm{\textsuperscript{1}}\textrm{pu}\textrm{\textsuperscript{7}}\textrm{(媳婦)}} & 2b & 連用+連用 & {\sffamily \textrm{果子ke}\textrm{\textsuperscript{2}}\textrm{tsi}\textrm{\textsuperscript{2}}\textrm{(水果)、報頭po}\textrm{\textsuperscript{3}}\textrm{hau}\textrm{\textsuperscript{5}}\textrm{(風暴的兆頭)}}\\
 1c & 連用+獨用 & {\sffamily \textrm{耳屎 hi}\textrm{\textsuperscript{7}}\textrm{sai}\textrm{\textsuperscript{2}}\textrm{、樓骹 lau}\textrm{\textsuperscript{5}}\textrm{kha}\textrm{\textsuperscript{1}} \textrm{(樓下)}} &  &  & \\
 1d & 連用+連用 & {\sffamily \textrm{會頭he}\textrm{\textsuperscript{7}}\textrm{thau}\textrm{\textsuperscript{5}}\textrm{、朋友ping}\textrm{\textsuperscript{5}}\textrm{iu}\textrm{\textsuperscript{2}}} &  &  & \\
\lspbottomrule
\end{tabularx}
\textrm{「複合」(compounding)如界定為「獨用形位」(free morpheme)的組合[FF0C?]則只能涵蓋(1a),其他(1b)(1c)(1d)都在排除之列。反之[FF0C?]複合界定為「實詞」(lexeme)的組合較為穩妥[FF0C?]則四種情況都能照顧到。派生的語根是實詞[FF0C?]可獨用也可連用。詞綴不帶詞彙意義。語言是變動不居的實體[FF0C?]形位可以從實詞轉變成虛詞。實詞過渡為虛詞可能有新舊並存的階段。「子」就是一個例子。「銃子」 tshing}\textrm{\textsuperscript{3}}\textrm{tsi}\textrm{\textsuperscript{2} }\textrm{的「子」意指粒狀物[FF0C?]算是實詞[FF0C?]但派生詞「果子」ke}\textrm{\textsuperscript{2}}\textrm{tsi}\textrm{\textsuperscript{2}}\textrm{泛指水果[FF0C?]包括像香蕉的非粒狀物;後綴「子」已失去指稱粒狀物的功能。「派生詞」(derivative)中「語幹」(stem)是加「詞綴」之前的狀態。沒有「詞綴」就沒有「語幹」。因此「語幹」可以是單純的「語幹」[FF0C?]只含「語根」不帶任何「詞綴」[FF0C?]或複雜的「語幹」[FF0C?]由帶「詞綴」的「語根」(root)組成。複雜的「語幹」中一個語根帶兩個以上的詞綴。詞綴先後順序的不同表示不同的語意[FF0C?]如「囡仔頭」gin}\textrm{\textsuperscript{2}}\textrm{{}-a}\textrm{\textsuperscript{2}} \textrm{thau}\textrm{\textsuperscript{5}} \textrm{(孩子王)}、\textrm{「椅頭仔」i}\textrm{\textsuperscript{2}}\textrm{thau}\textrm{\textsuperscript{5}}\textrm{a}\textrm{\textsuperscript{2}}\textrm{(凳子)}、\textrm{前者是複雜的複合詞[FF0C?]第一個成分派生詞由連用語根「囡」gin}\textrm{\textsuperscript{2}}\textrm{和後綴 {}-a}\textrm{\textsuperscript{2}}\textrm{「仔」組合形成獨用的語詞「囡仔」gin}\textrm{\textsuperscript{2}}\textrm{{}-a}\textrm{\textsuperscript{2}} \textrm{(孩子) (即實詞)}[FF0C?]\textrm{再和另一个連用的實詞「頭」thau}\textrm{\textsuperscript{5}}\textrm{thau}\textrm{\textsuperscript{5} }\textrm{(領袖)組成複雜的複合詞「囡仔頭」。後者是複雜的派生詞[FF0C?]連用的語根「椅」和後綴「頭」新組成派生詞「椅頭」[FF0C?]然後「椅頭」再加上後綴「仔」[FF0C?]造就複雜派生詞「椅頭仔」。就後綴「仔」而言[FF0C?]「椅頭仔」是語根加後綴的語幹。複雜的派生詞或複合詞[FF0C?]分階段的組合有一定的作業程序[FF0C?]運作順序不能顛倒[FF0C?]否則會產生不合語法的後果。(以上的論述參閱連 2000)}

\subsection{\rmfamily 2.7 從複合到派生}

\textrm{複合詞是由實詞組成的。假若其中一個實詞逐漸虛化成詞綴[FF0C?]就形成了實詞加詞綴的派生詞了。以下以「頭」為例子來說明複合詞變為派生詞的演變。}

\subsubsection{\rmfamily 2.7.1 「頭」的多重功能}

\textrm{漢語(台語也不例外)中實詞與虛形位(即詞綴)多半沒有形式的區分[FF0C?]但是可以經過結構的分析和音義的關係找出兩者的區分來。從下列AB兩個表[FF0C?]可以看出「頭」thau}\textrm{\textsuperscript{5}}\textrm{語音形式不變[FF0C?]可以當實詞「頭1」和詞綴「頭2」使用[FF1A?]}

\textrm{複合詞中實詞「頭1」的語意延伸}

表2-2 複合詞中的實詞

\tablefirsthead{}

\tabletail{}
\tablelasttail{}
\begin{tabularx}{\textwidth}{XXX}
\lsptoprule

A & 複合詞中的實詞 & 例子\\
\hhline{--~}
頭1 & \multicolumn{1}{X}{ 語意內涵} & \\
1 & 人的頭部 & {\sffamily \textrm{儂lang}\textrm{\textsuperscript{5}}\textrm{頭}}\\
2 & 動物的頭部 & {\sffamily \textrm{羊iunn}\textrm{\textsuperscript{5}}\textrm{頭、蛇tsua}\textrm{\textsuperscript{5}}\textrm{頭}}\\
3 & 垂直物體的頂部 & {\sffamily \textrm{山suann}\textrm{\textsuperscript{1}}\textrm{頭、褲khoo}\textrm{\textsuperscript{3}}\textrm{頭、裙kun}\textrm{\textsuperscript{5}}\textrm{頭}}\\
4 & 垂直物體的根部 & {\sffamily \textrm{樹tshiu}\textrm{\textsuperscript{7}}\textrm{頭、竹tik}\textrm{\textsuperscript{4}}\textrm{頭}}\\
5 & 事物的源頭 & {\sffamily \textrm{溪khue}\textrm{\textsuperscript{1}}\textrm{頭、水tsui}\textrm{\textsuperscript{2}}\textrm{頭、圳tsun}\textrm{\textsuperscript{3}}\textrm{頭、風hong}\textrm{\textsuperscript{1}}\textrm{頭}}\\
6 & 水平物體的一端 & {\sffamily \textrm{桌toh}\textrm{\textsuperscript{4}}\textrm{頭、街kue}\textrm{\textsuperscript{1}}\textrm{頭、路loo}\textrm{\textsuperscript{7}}\textrm{頭}}\\
7 & 時間的開頭部分 & {\sffamily \textrm{年ni}\textrm{\textsuperscript{5}}\textrm{頭、冬tang}\textrm{\textsuperscript{1}}\textrm{頭、月geh}\textrm{\textsuperscript{8}}\textrm{頭}}\\
8 & 事件開始的部分 & {\sffamily \textrm{藥ioh}\textrm{\textsuperscript{8}}\textrm{頭、果子ke}\textrm{\textsuperscript{2}}\textrm{{}-tsi}\textrm{\textsuperscript{2}}\textrm{頭、奶lin}\textrm{\textsuperscript{1}}\textrm{頭}}\\
9 & 群體的領袖/活動的組織者 & {\sffamily \textrm{工kang}\textrm{\textsuperscript{1}}\textrm{頭、乞丐khit}\textrm{\textsuperscript{4}}\textrm{+tsiah}\textrm{\textsuperscript{8}}\textrm{頭、囡仔gin}\textrm{\textsuperscript{2}}\textrm{{}-a}\textrm{\textsuperscript{2}}\textrm{頭、會he}\textrm{\textsuperscript{7}}\textrm{頭、繳kiau}\textrm{\textsuperscript{2}}\textrm{頭、組tsoo}\textrm{\textsuperscript{2}}\textrm{頭}}\\
10 & 事物殘餘的部分 & {\sffamily \textrm{蕃薯han}\textrm{\textsuperscript{1}}\textrm{+tsu}\textrm{\textsuperscript{5}}\textrm{頭 、豆tau}\textrm{\textsuperscript{7}}\textrm{頭、磚仔tsng}\textrm{\textsuperscript{7}}\textrm{{}-a}\textrm{\textsuperscript{2}}\textrm{頭}}\\
\lspbottomrule
\end{tabularx}
\textrm{「頭1」是人體的重要部分[FF0C?]人類的認知思維活動[FF0C?]其發展有一定的運作順序[FF0C?]準此[FF0C?]以人類自身的頭部為起點由近而遠[FF0C?]推及動物[FF0C?]再及於植物[FF0C?]從有生命的到無生命的實體[FF0C?]從具像的空間關係推展到抽象的時間關係[FF0C?]從具體熟知的到抽象生疏的對象。依次拓展[FF0C?]將這一系列的對象串連起來。其間的核心語意都能透過轉喻和隱喻串連起來。這點可以從每類的註解中體會出來。}

\textrm{第4組中「樹頭」}是指樹的根部[FF0C?]但在俗語\textrm{中「食果子拜樹頭」的「樹頭」是樹木成長的開端和源頭[FF0C?]我們享受結出的果實[FF0C?]就像飲水思源一樣需感謝樹木的根源。}因此\textrm{「樹頭」}的隱喻用法也可歸為第5組。第8組的\textrm{「頭」}是指事件的開端[FF0C?]如的\textrm{「}藥\textrm{頭」}指藥材泡煮出的頭一劑量[FF0C?]\textrm{「}果子\textrm{頭」}指水果採收的頭一批[FF0C?]\textrm{「}奶\textrm{頭」}指母奶擠出的頭一份。\textrm{第9組是以「隱喻」(metaphor)的方式表示群體的核心[FF0C?]就像人的頭部那麼重要。第10組的「蕃薯han}\textrm{\textsuperscript{1}}\textrm{tsu}\textrm{\textsuperscript{5}}\textrm{頭」是採收蕃薯時所留下的殘餘的部分。}以上未特別說明的各組可以從釋義中推敲出來。

\textrm{派生詞中的詞綴「頭2」}

\rmfamily
表2-3 派生詞中的詞綴

\tablefirsthead{}

\tabletail{}
\tablelasttail{}
\begin{tabularx}{\textwidth}{XXX}
\lsptoprule

B & {\sffamily \textrm{派生詞中的詞綴}} & 例子\\
\hhline{--~}
頭2 & \multicolumn{1}{X}{{\sffamily \textrm{語意、語法內涵}}} & \\
1 & 物體的全域 & {\sffamily \textrm{路loo}\textrm{\textsuperscript{7}}\textrm{頭}\textrm{\textsubscript{b}}}\\
2 & 表示方位 & {\sffamily \textrm{頂ting}\textrm{\textsuperscript{2}}\textrm{頭、下e}\textrm{\textsuperscript{7}}\textrm{頭、內lai}\textrm{\textsuperscript{7}}\textrm{頭、外gua}\textrm{\textsuperscript{7}}\textrm{頭、此tsit}\textrm{\textsuperscript{4}}\textrm{頭、彼hit}\textrm{\textsuperscript{4}}\textrm{頭、後au}\textrm{\textsuperscript{7}}\textrm{頭、車tshia}\textrm{\textsuperscript{1}}\textrm{頭、埠poo}\textrm{\textsuperscript{1}}\textrm{頭}}\\
3 & {\sffamily \textrm{事物的外觀、狀態}} & {\sffamily \textrm{看khuann}\textrm{\textsuperscript{3}}\textrm{頭、派phai}\textrm{\textsuperscript{3}}\textrm{頭、湯thng}\textrm{\textsuperscript{1}}\textrm{頭、症tsing}\textrm{\textsuperscript{3}}\textrm{頭}}\\
4 & {\sffamily \textrm{表示份量、程度}} & {\sffamily \textrm{重tang}\textrm{\textsuperscript{7}}\textrm{頭、秤tshin}\textrm{\textsuperscript{3}}\textrm{頭、芳phang}\textrm{\textsuperscript{1}}\textrm{頭、擔tann}\textrm{\textsuperscript{3}}\textrm{頭}}

{\sffamily 、\textrm{當tong}\textrm{\textsuperscript{3}}\textrm{頭、勢se}\textrm{\textsuperscript{3}}\textrm{頭、力lat}\textrm{\textsuperscript{8}}\textrm{頭、尺tshioh}\textrm{\textsuperscript{4}}\textrm{頭、膽tann}\textrm{\textsuperscript{2}}\textrm{頭、粒liap}\textrm{\textsuperscript{8}}\textrm{頭}}\\
5 & 表示事件 & {\sffamily \textrm{穡sik}\textrm{\textsuperscript{4}}\textrm{頭、齣tshut}\textrm{\textsuperscript{8}}\textrm{頭、報po}\textrm{\textsuperscript{3}}\textrm{頭、空khang}\textrm{\textsuperscript{1}}\textrm{頭}}\\
6 & 特定的動作 & {\sffamily \textrm{看khuann}\textrm{\textsuperscript{3}}\textrm{頭、拳kun}\textrm{\textsuperscript{5}}\textrm{頭、塞tsat}\textrm{\textsuperscript{8}}\textrm{頭}}\\
7 & 構形的空形位 & {\sffamily \textrm{石tsioh}\textrm{\textsuperscript{8}}\textrm{頭、枕tsim}\textrm{\textsuperscript{2}}\textrm{頭、日jit}\textrm{\textsuperscript{8}}\textrm{頭、罐kuan}\textrm{\textsuperscript{3}}\textrm{頭、骨kut}\textrm{\textsuperscript{4}}\textrm{頭}\textrm{\textsubscript{b}}\textrm{、鏡kiann}\textrm{\textsuperscript{3}}\textrm{頭}}\\
\lspbottomrule
\end{tabularx}
\textrm{從左列的釋義中可以窺見各組的語義語法特徵。其共通的特徵都是不能用來指稱個體[FF0C?]而是指稱全域、方位、外觀、狀態、分量程度、事件動作[FF0C?]甚至詞彙意義完全消失。另外[FF0C?]以下幾點需加說明。第1組「車tshia}\textrm{\textsuperscript{1}}\textrm{頭」指車站[FF0C?]即火車停靠的車站[FF0C?]應算轉喻。第7組「枕tsim}\textrm{\textsuperscript{2}}\textrm{頭」的已經沒有詞彙語義[FF0C?]鏡kiann}\textrm{\textsuperscript{3}}\textrm{頭也是如此[FF0C?]但受到中文的影響[FF0C?]「鏡頭」也指影像[FF0C?]畫面[FF0C?]如這麼解釋[FF0C?]也可看成一種「換喻」(metonymy)。「骨kut}\textrm{\textsuperscript{4}}\textrm{頭」甚至以轉化為表示強調的情緒用語[FF0C?]如「甚麼死人骨頭个代誌」。總之[FF0C?]詞綴「頭2」顯現詞彙意義程度深淺的削弱或沖淡[FF0C?]取而代之的是語法功能的增強。(本節參閱連1999)}

\subsubsection{\rmfamily 2.7.2 「仔」的三種形態學功能}

\textrm{「仔」詞源一般認定為「囝」kiann}\textrm{\textsuperscript{2}}\textrm{(*<kian}\textrm{\textsuperscript{2}}\textrm{) (中文念作jian}\textrm{\textsuperscript{3}}\textrm{)[FF0C?]「仔」這個俗體字是訓讀[FF0C?]即只取其義不取其音。這個詞可能借自「南亞」(Austroasiatic)語。(詳 \citealt{NormanMei1976})「囝」kiann}\textrm{\textsuperscript{2}}\textrm{原是意指小孩的實詞[FF0C?]後來伴隨虛化[FF0C?]語音形式也縮約弱化為a}\textrm{\textsuperscript{2}}。\textrm{現代台語中實詞「囝」kiann}\textrm{\textsuperscript{2}}\textrm{和形位「仔」a}\textrm{\textsuperscript{2}}\textrm{並存。「仔」a}\textrm{\textsuperscript{2}}\textrm{有三種形態功能}[FF1A?]\textrm{(1)辨意功能、(2)轉換詞類的功能和(3)純造詞功能。「囝」由原指小孩[FF0C?]虛化為「小稱後綴」(diminutive suffix)}[FF0C?]\textrm{再進一步虛化為不帶詞彙意義的形位。現代台語「雞仔」kue}\textrm{\textsuperscript{1}}\textrm{a}\textrm{\textsuperscript{2}}\textrm{泛指雞[FF0C?]「雞仔囝」kue}\textrm{\textsuperscript{1}}\textrm{a}\textrm{\textsuperscript{2}} \textrm{kiann}\textrm{\textsuperscript{2}}\textrm{才是小雞。}

\paragraph{2.7.2.1 辨意功能}

\textrm{一般把後綴「仔」稱做小稱後綴[FF0C?]可是有些情況帶後綴「仔」的派生詞並沒有小的意思。如下例所示[FF0C?]「仔」本身不具詞彙意義[FF0C?]但是可以用來區辨同詞類的語詞。}

\rmfamily
表2-4純辨意功能的詞綴

\tablefirsthead{}

\tabletail{}
\tablelasttail{}
\begin{tabularx}{\textwidth}{XXXX}
\lsptoprule

 例子 & 釋意 & 例子 & 釋意\\
{\sffamily \textrm{米 bi}\textrm{\textsuperscript{2}}} &  & {\sffamily \textrm{糖 thng}\textrm{\textsuperscript{5}}} & \\
{\sffamily \textrm{米仔 bi}\textrm{\textsuperscript{2}} \textrm{a}\textrm{\textsuperscript{2}}} & “爆米花” & {\sffamily \textrm{糖仔 thng}\textrm{\textsuperscript{5}}} & “糖果”\\
{\sffamily \textrm{霜 sng}\textrm{\textsuperscript{1}}} &  & {\sffamily \textrm{會 he}\textrm{\textsuperscript{7}}} & \\
{\sffamily \textrm{霜仔 sng}\textrm{\textsuperscript{1}} \textrm{a}\textrm{\textsuperscript{2}}} & “刨冰” & {\sffamily \textrm{會仔 he}\textrm{\textsuperscript{7}} \textrm{a}\textrm{\textsuperscript{2}}} & “互助會”\\
{\sffamily \textrm{茶店 te}\textrm{\textsuperscript{5}} \textrm{tiam}\textrm{\textsuperscript{3}}} & “茶館” & {\sffamily \textrm{車 tshia}\textrm{\textsuperscript{1}}} & \\
{\sffamily \textrm{茶店仔te}\textrm{\textsuperscript{5}} \textrm{tiam}\textrm{\textsuperscript{3}} \textrm{a}\textrm{\textsuperscript{2}}} & “茶室” & {\sffamily \textrm{車仔 tshia}\textrm{\textsuperscript{1}} \textrm{a}\textrm{\textsuperscript{2}}} & “裁縫機”\\
\lspbottomrule
\end{tabularx}
\textrm{或許有人會問[FF0C?]「音位」(phoneme)也是不帶語義但可以區別語詞的單位[FF0C?]如此一來怎麼區分音位和形位?此問題的關鍵在於[FF0C?]音位是「音韻」(phonology)的基本單位[FF0C?]形位是「形態」(morphology)的基本單位。有些「純形位」(morphome)只具備造詞或區別語詞的功能。}

\paragraph{2.7.2.2 轉換詞類的功能}

\textrm{後綴「仔」a}\textrm{\textsuperscript{2}}\textrm{的主要造詞功能是將某詞類派生為新的詞類。以下列舉最常見的類型[FF0C?]其中由重疊形容詞和重疊動詞派生出來的副詞似乎最有「孳生力」(productivity)}。

\rmfamily
表2-5 造新詞的後綴「仔」

\tablefirsthead{}

\tabletail{}
\tablelasttail{}
\begin{tabularx}{\textwidth}{XXXX}
\lsptoprule

 原詞類 & 轉入的詞類 & 例子 & 釋意\\
動詞 & 名詞 & {\sffamily \textrm{塞(that}\textrm{\textsuperscript{4}}\textrm{)仔}} & “塞子”\\
\hhline{~~--} &  & {\sffamily \textrm{刺(tshi}\textrm{\textsuperscript{3}}\textrm{)仔}} & “有刺植物”\\
形容詞 & 名詞 & {\sffamily \textrm{湳(lam}\textrm{\textsuperscript{3}}\textrm{)仔}} & “深的溼田”\\
\hhline{~~--} &  & {\sffamily \textrm{圓(inn}\textrm{\textsuperscript{5}}\textrm{)仔}} & “湯圓”\\
\hhline{~~--} &  & {\sffamily \textrm{崎(kia}\textrm{\textsuperscript{7}}\textrm{)仔}} & “斜坡”\\
反意形容詞 & 名詞 & {\sffamily \textrm{寒熱(kiann}\textrm{\textsuperscript{5}} \textrm{jiet}\textrm{\textsuperscript{8}}\textrm{)仔}} & “瘧疾”\\
重疊形容詞 & 副詞 & {\sffamily \textrm{慢慢(ban}\textrm{\textsuperscript{7}}\textrm{ban}\textrm{\textsuperscript{7}}\textrm{)仔}} & “慢慢的”\\
重疊動詞 & 副詞 & {\sffamily \textrm{笑笑(tshio}\textrm{\textsuperscript{3}}\textrm{tsio}\textrm{\textsuperscript{3}}\textrm{)仔}} & “微笑的”\\
古代副詞 & 副詞 & {\sffamily \textrm{嶄然(tsiam}\textrm{\textsuperscript{2}} \textrm{jian}\textrm{\textsuperscript{5}}\textrm{)仔}} & “非常”\\
動詞組 & 名詞 & {\sffamily \textrm{治/刣(thai}\textrm{\textsuperscript{5}}\textrm{)豬仔}} & “殺豬的”\\
\lspbottomrule
\end{tabularx}
\textrm{依這種分析方式「仔」的派生作用是將原詞轉變為新的詞類。另一種可行的方案是假定語根沒有詞性[FF0C?]「仔」的作用是賦予語根詞性。}

\paragraph{2.7.2.3 純造詞功能}

\textrm{後綴「仔」a}\textrm{\textsuperscript{2}}\textrm{有時只是純形式的形位[FF0C?]只有造詞的功能沒有明顯的詞彙意義。這裡至少有兩個情況[FF0C?]如下圖所示[FF1A?]}

\tablefirsthead{}

\tabletail{}
\tablelasttail{}
\begin{tabularx}{\textwidth}{XXXX} & 語根 & 不加“仔”的例子 & 加“仔”的例子\\
\lsptoprule
1. & 獨用 & {\sffamily \textrm{鴨ah}\textrm{\textsuperscript{4}}\textrm{、雞kue}\textrm{\textsuperscript{1}}} & {\sffamily \textrm{鴨仔ah}\textrm{\textsuperscript{4}} \textrm{a}\textrm{\textsuperscript{2}}\textrm{、雞仔 kue}\textrm{\textsuperscript{1}}\textrm{a}\textrm{\textsuperscript{2}}}\\
2. & 不獨用 & {\sffamily \textrm{*囡gin}\textrm{\textsuperscript{2}}\textrm{、*尪ang}\textrm{\textsuperscript{1}}} & {\sffamily \textrm{囡仔 gin}\textrm{\textsuperscript{2}} \textrm{a}\textrm{\textsuperscript{2}}\textrm{、尪仔 ang}\textrm{\textsuperscript{1}} \textrm{a}\textrm{\textsuperscript{2}}}\\
\lspbottomrule
\end{tabularx}
\textrm{(1)語根為獨用形位[FF0C?]加「仔」和不加「仔」都可以[FF0C?]帶「仔」並不明顯表示小的意思。(2)語根為連用形位(即不能單獨出現)}[FF0C?]\textrm{非加「仔」不可。這兩個情況「仔」都是純形位。「尪」ang}\textrm{\textsuperscript{1}}\textrm{和同音詞「翁」ang}\textrm{\textsuperscript{1}}\textrm{不是同一個詞。「翁」ang}\textrm{\textsuperscript{1}}\textrm{用於「翁婿」ang}\textrm{\textsuperscript{1}}\textrm{sai}\textrm{\textsuperscript{3}}\textrm{(丈夫)或「翁仔某」ang}\textrm{\textsuperscript{1} }\textrm{a}\textrm{\textsuperscript{2}} \textrm{boo}\textrm{\textsuperscript{2}}\textrm{(<「翁合某」ang}\textrm{\textsuperscript{1}} \textrm{kah}\textrm{\textsuperscript{4}} \textrm{boo}\textrm{\textsuperscript{2}}\textrm{)(夫妻)}。\textrm{「尪」原意可能是圖像之意[FF0C?]用於「尪仔冊」ang}\textrm{\textsuperscript{1} }\textrm{a}\textrm{\textsuperscript{2}}\textrm{tsheh}\textrm{\textsuperscript{4}} \textrm{(連環畫冊)},\textrm{「尪仔物」ang}\textrm{\textsuperscript{1} }\textrm{a}\textrm{\textsuperscript{2} }\textrm{mih}\textrm{\textsuperscript{8}} \textrm{(玩偶),「布袋戲尪仔」poo}\textrm{\textsuperscript{3}}\textrm{te}\textrm{\textsuperscript{7}} \textrm{hi}\textrm{\textsuperscript{3}} \textrm{ang}\textrm{\textsuperscript{1}}\textrm{a}\textrm{\textsuperscript{2} }\textrm{(布袋戲偶)}、\textrm{「尪仔標」ang}\textrm{\textsuperscript{1}}\textrm{a}\textrm{\textsuperscript{2}} \textrm{phiau}\textrm{\textsuperscript{1}} \textrm{(有圖像的圓形紙牌[FF0C?]供小孩玩耍)}。\textrm{「尪」ang}\textrm{\textsuperscript{1}}\textrm{不能獨用[FF0C?]「尪仔」ang}\textrm{\textsuperscript{1} }\textrm{a}\textrm{\textsuperscript{2}}\textrm{指圖畫或玩偶[FF0C?]如「畫尪仔」ui}\textrm{\textsuperscript{7}} \textrm{ang}\textrm{\textsuperscript{1}}\textrm{a}\textrm{\textsuperscript{2}}\textrm{(畫畫兒)}、\textrm{「布尪仔」poo}\textrm{\textsuperscript{3}} \textrm{ang}\textrm{\textsuperscript{1}}\textrm{a}\textrm{\textsuperscript{2}}\textrm{(玩偶)}。

\textrm{一般討論小稱後綴時[FF0C?]常以外界客觀事物的大小作為是否可以加小稱後綴的標準。只有指涉小的物件的語根才能接小稱後綴。但是這樣的設想有它說不通的地方。首先[FF0C?]大小是相對的[FF1A?]象比老虎大[FF0C?]老虎比狗大[FF0C?]狗比雞大[FF0C?]雞比螞蟻大。但是並非所有大的對象就不能使用「仔」當後綴[FF0C?]比如[FF0C?]老虎和獅子差不多一樣大[FF0C?]可是「虎」hoo}\textrm{\textsuperscript{2}}\textrm{可以加「仔」[FF0C?]「獅」sai}\textrm{\textsuperscript{1}}\textrm{不能「獅仔」被同音詞「師仔」(學徒)阻絕掉之故。反過來說[FF0C?]蟲和蚯蚓差不多一樣大[FF0C?]「蟲」thang}\textrm{\textsuperscript{5}}\textrm{不能加後綴「仔」[FF0C?]「土蚓」too}\textrm{\textsuperscript{5}} \textrm{un}\textrm{\textsuperscript{2}}\textrm{可以。}

\textrm{語意的演變通常不是一刀兩斷的[FF0C?]而是拖泥帶水的。常常有所謂的雙棲詞[FF0C?]即一個詞同時涵蓋實意和虛意。“囝”kiann}\textrm{\textsuperscript{2}}\textrm{就是一個好的例子[FF0C?]如下表所示[FF1A?]}

表2-6 雙棲詞

\tablefirsthead{}

\tabletail{}
\tablelasttail{}
\begin{tabularx}{\textwidth}{XXX}
\lsptoprule

 例子 & a.釋意 & b.釋意\\
{\sffamily \textrm{1.豬仔囝 ti}\textrm{\textsuperscript{1}} \textrm{a}\textrm{\textsuperscript{2}} \textrm{kiann}\textrm{\textsuperscript{2}}} & “豬生的兒子” & “小豬”\\
{\sffamily \textrm{2.查甫囝 tsa}\textrm{\textsuperscript{1} }\textrm{poo}\textrm{\textsuperscript{1}} \textrm{kiann}\textrm{\textsuperscript{2}}} & “兒子” & “男子漢”\\
{\sffamily \textrm{3.囡仔囝 gin}\textrm{\textsuperscript{2}}\textrm{{}-a}\textrm{\textsuperscript{2}} \textrm{kiann}\textrm{\textsuperscript{2}}} &  & “小孩”\\
\lspbottomrule
\end{tabularx}
\textrm{「囝」kiann}\textrm{\textsuperscript{2}}\textrm{單用時通常專指“兒子”}[FF0C?]\textrm{但在複合詞中[FF0C?]如(1)和(2)}[FF0C?]\textrm{兼有實意(a.)和虛意(b.)}。\textrm{「查甫囝」也可指男子漢[FF0C?] 如「查甫囝敢做敢當」 (男子漢敢做敢當)。相形之下[FF0C?](3)裡頭的「囝」kiann}\textrm{\textsuperscript{2}}\textrm{已經虛化[FF0C?]不帶實意[FF0C?]即「囡仔囝」gin}\textrm{\textsuperscript{2}}\textrm{a}\textrm{\textsuperscript{2}} \textrm{kiann}\textrm{\textsuperscript{2}}\textrm{和「囡仔」gin}\textrm{\textsuperscript{2}}\textrm{{}-a}\textrm{\textsuperscript{2}}\textrm{語意沒有什麼不同。總之[FF0C?]從上面的討論中可以看出[FF0C?]「囝」kiann}\textrm{\textsuperscript{2}}\textrm{還有兩棲特性[FF0C?]因此它的虛化還未完成[FF0C?]而「囝」a}\textrm{\textsuperscript{2}}\textrm{已完成了。(本節參閱連1998)}

\subsection{\rmfamily 2.8 形態和語言層次的互動}

\textrm{台語在歷史的演化中[FF0C?]累積了豐富的文白讀層次[FF0C?]不只反映在語音還顯現於詞法的差異。「老」可分成三個層次: lau}\textrm{\textsuperscript{7}} \textrm{{\textasciitilde} lau}\textrm{\textsuperscript{2}} \textrm{{\textasciitilde} lo}\textrm{\textsuperscript{2}}[FF0C?]\textrm{下表中整理出三個不同的層次。}

 \textrm{「老」}的文白讀區分

\tablefirsthead{}

\tabletail{}
\tablelasttail{}
\begin{tabularx}{\textwidth}{XXXX}
\lsptoprule

 層次 & 文讀層1 & 文讀層2 & 白讀層\\
 老 & {\sffamily \textrm{lo}\textrm{\textsuperscript{2}}} & {\sffamily \textrm{lau}\textrm{\textsuperscript{2}}} & {\sffamily \textrm{lau}\textrm{\textsuperscript{7}}}\\
\lspbottomrule
\end{tabularx}
\rmfamily
下列圖表中分別整理出各個不同層次“老”的體現

 \textrm{「老」lo}\textrm{\textsuperscript{2}}的用法

\tablefirsthead{}

\tabletail{}
\tablelasttail{}
\begin{tabularx}{\textwidth}{XX}
\lsptoprule

{\sffamily \textrm{lo}\textrm{\textsuperscript{2}} \textrm{老}} & 層次組成\\
{\sffamily \textrm{lo}\textrm{\textsuperscript{2}}\textrm{{}-hu}\textrm{\textsuperscript{1}} \textrm{老夫}} & 文讀+文讀\\
{\sffamily \textrm{lo}\textrm{\textsuperscript{2}}\textrm{{}-ia}\textrm{\textsuperscript{5}} \textrm{老爺}} & 文讀+文讀\\
{\sffamily \textrm{lo}\textrm{\textsuperscript{2}}\textrm{{}-ong}\textrm{\textsuperscript{1}} \textrm{老翁}} & 文讀+文讀\\
{\sffamily \textrm{o}\textrm{\textsuperscript{1}}\textrm{{}-lo}\textrm{\textsuperscript{2}} \textrm{呵咾} } & ?文讀+文讀\\
\lspbottomrule
\end{tabularx}
 \textrm{「老」lau}\textrm{\textsuperscript{2}}的用法

\tablefirsthead{}

\tabletail{}
\tablelasttail{}
\begin{tabularx}{\textwidth}{XX}
\lsptoprule

{\sffamily \textrm{老 lau}\textrm{\textsuperscript{2}}} & 層次組成\\
{\sffamily \textrm{lau}\textrm{\textsuperscript{2}}\textrm{{}-tua}\textrm{\textsuperscript{7}} \textrm{老大}} & 文讀+白讀\\
{\sffamily \textrm{lau}\textrm{\textsuperscript{2}}\textrm{{}-sue}\textrm{\textsuperscript{7}} \textrm{老細}} & 文讀+白讀\\
{\sffamily \textrm{lau}\textrm{\textsuperscript{2}}\textrm{{}-hiann}\textrm{\textsuperscript{1}} \textrm{老兄}} & 文讀+白讀\\
{\sffamily \textrm{lau}\textrm{\textsuperscript{2}}\textrm{{}-li}\textrm{\textsuperscript{2}}\textrm{{}-e}\textrm{\textsuperscript{1}} \textrm{老李个}} & 文讀+白讀\\
\lspbottomrule
\end{tabularx}
 \textrm{「老」lau}\textrm{\textsuperscript{7}}的用法

\tablefirsthead{}

\tabletail{}
\tablelasttail{}
\begin{tabularx}{\textwidth}{XX}
\lsptoprule

{\sffamily \textrm{lau}\textrm{\textsuperscript{7}} \textrm{老}} & 層次組成\\
{\sffamily \textrm{lau}\textrm{\textsuperscript{7}}\textrm{{}-tua}\textrm{\textsuperscript{7}} \textrm{老大儂}} & 白讀+白讀\\
{\sffamily \textrm{lau}\textrm{\textsuperscript{7}}\textrm{{}-bu}\textrm{\textsuperscript{2}} \textrm{老母}} & 白讀+白讀\\
{\sffamily \textrm{lau}\textrm{\textsuperscript{7}}\textrm{{}-pe}\textrm{\textsuperscript{7}} \textrm{老父}} & 白讀+白讀\\
{\sffamily \textrm{lau}\textrm{\textsuperscript{7}}\textrm{{}-po}\textrm{\textsuperscript{5}} \textrm{老婆}} & 白讀+白讀\\
\lspbottomrule
\end{tabularx}
\textrm{「老」的兩個文讀層lo}\textrm{\textsuperscript{2}} \textrm{(又讀為noo}\textrm{\textsuperscript{2}}\textrm{)和lau}\textrm{\textsuperscript{2}}\textrm{都讀陰上[FF0C?]前者是舊文讀[FF0C?]後者是新文讀[FF0C?]借入閩南語有時間先後之別。前者只能用於戲文中的對話[FF0C?]後者可以用於現代的口語的對話中[FF0C?]但總帶有北方官話移入的色彩。不是台語固有的表現法。「呵咾」o}\textrm{\textsuperscript{1}}\textrm{{}-lo}\textrm{\textsuperscript{2}}\textrm{(讚美)是複合詞[FF0C?]其中「呵」o}\textrm{\textsuperscript{1}}\textrm{(也寫作「謳)」)是讚美之意。至於「咾」是否與「老」同源待考。「老」lau}\textrm{\textsuperscript{2}}\textrm{除前綴外也可充當獨用的形容詞[FF0C?]失去了原意發展出兩種隱喻的意義: \REF{ex:key:1}擅長[FF0C?]如「伊英語足老个」i}\textrm{\textsuperscript{1}} \textrm{ing}\textrm{\textsuperscript{1}}\textrm{{}-gu}\textrm{\textsuperscript{2}} \textrm{tsiok}\textrm{\textsuperscript{4}} \textrm{lau}\textrm{\textsuperscript{2}} \textrm{e}\textrm{\textsuperscript{1}} \textrm{(他的英文很溜)}[FF0C?]\textrm{(2) (食物等)變壞[FF0C?]如「果子老了了」ke}\textrm{\textsuperscript{2}}\textrm{{}-tsi}\textrm{\textsuperscript{2}} \textrm{lau}\textrm{\textsuperscript{2}} \textrm{liau}\textrm{\textsuperscript{2}} \textrm{liau}\textrm{\textsuperscript{2}} \textrm{(水果都壞掉了)}。\textrm{文讀的lau}\textrm{\textsuperscript{2}}\textrm{也語法化成前綴[FF0C?]如「老大」lau}\textrm{\textsuperscript{2}}\textrm{{}-tua}\textrm{\textsuperscript{7}} \textrm{(兄弟間年紀較大的)}、\textrm{「老細」lau}\textrm{\textsuperscript{2}}\textrm{{}-sue}\textrm{\textsuperscript{7}} \textrm{(兄弟間較年輕的)[FF0C?]這二個例子中‘老’的原意已經不見了。}

\textrm{讀為陽去的「老」lau}\textrm{\textsuperscript{7}}\textrm{才是貨真價實的本土固有語詞。白讀的lau}\textrm{\textsuperscript{7}}\textrm{仍保留其字面上「老」的意思[FF0C?]與「少年」siau}\textrm{\textsuperscript{3}}\textrm{{}-lian}\textrm{\textsuperscript{5}}\textrm{(年輕)相反。lau}\textrm{\textsuperscript{7}}\textrm{雖然也發展出委婉的意思[FF0C?]如「老去啊」lau}\textrm{\textsuperscript{7}} \textrm{khi}\textrm{\textsuperscript{3}} \textrm{a}\textrm{\textsuperscript{1} }\textrm{(過世了)}[FF0C?]\textrm{但跟原意還是有緊密的關係。中文的「老」有表頻率的時間副詞用法[FF0C?]如「他老遲到」[FF0C?]但台語不論文白讀都沒有這種副詞的用法[FF0C?]可見「老」的這種副詞的用法是相當晚起的發展。不過「老」lau}\textrm{\textsuperscript{7}}\textrm{倒是有「老朋友」的用法[FF0C?]表示久遠的友誼關係[FF0C?]但未知是否為移植自中文。(本節參閱\citealt{Lien2001})}

\subsection{\rmfamily 2.9 台語鏈接詞‘个’的派生功能}

\textrm{「个」(可能源自上古漢語的指示詞「其」[FF0C?]「个」是一般文獻中的俗寫法)基本上用來表示修飾關係[FF0C?]即X个Y的結構上X修飾Y}。\textrm{Y是「隨意」(optional)成分。這種結構是形態部門中的一種造詞方式。通過「个」這樣的「鏈接」(linker)來造詞。「个」也可看成名物化的標記(nominalizer)。以下依次介紹「个」的派生功能。}

\begin{listWWviiiNumxivleveli}
\item \begin{styleqwerty}\rmfamily
表領屬關係或限定關係
\end{styleqwerty}
\end{listWWviiiNumxivleveli}

\begin{listWWviiiNumxxleveli}
\item \begin{styleqwerty}\rmfamily
領屬關係
\end{styleqwerty}
\end{listWWviiiNumxxleveli}

\textrm{我个冊(我的書)}、\textrm{伊个家伙(他的財產)}、\textrm{桌上个稿紙(桌上的稿紙)}

\textrm{我/你/伊个(我/你/他的)}、\textrm{阮/咱个(我們/咱們的)}、\textrm{伊/因个(他/他們的)}

\begin{listWWviiiNumxxleveli}
\item \begin{styleqwerty}\rmfamily
限定關係
\end{styleqwerty}
\end{listWWviiiNumxxleveli}
\rmfamily
鏈接出現於定語詞組中類別詞和名詞之間

\textrm{此款个代誌(這種事情)}、\textrm{此款个話 (這種話)}、\textrm{種種个事業(種種的事業)}

\begin{listWWviiiNumxivleveli}
\item \begin{styleqwerty}\rmfamily
外元充當中心語
\end{styleqwerty}
\end{listWWviiiNumxivleveli}

\textrm{以下的例子都是由單句派生出來的。比如含二元謂語的句子[FF0C?]以外元為焦點[FF0C?]提取為中心語。如「某人賣膏藥」中外元為「某人」作為謂語「某人賣膏藥」的修飾對象[FF0C?]派生出「賣膏藥个(人)」}。

\textrm{賣膏藥个(賣膏藥的)}、\textrm{看日个(擇日的)}、\textrm{看地理个(看風水的)}、\textrm{收數个(收帳的)}、\textrm{顧倉庫个(管倉庫的)}、\textrm{教書个(教書的)}

\rmfamily
其他類似的例子

\textrm{有志氣个人(有志氣的人)}、\textrm{無天良个人(無天良的人)}、\textrm{無情無義个人(無情無義个的人)}、\textrm{做父母个人(做父母的人)}、\textrm{做小弟个人(做弟弟的人)}

\begin{listWWviiiNumxivleveli}
\item \begin{styleqwerty}\rmfamily
內元充當中心語
\end{styleqwerty}
\end{listWWviiiNumxivleveli}
\rmfamily
二元的謂語的句子中內元成為派生詞的中心語

\textrm{拾著个物(撿到的東西)}、\textrm{偷提个物(偷拿的東西)}、\textrm{卜用个物(要用的東西)}、\textrm{食穿个物(吃的穿的(的)東西)}

\begin{listWWviiiNumxivleveli}
\item \begin{styleqwerty}\rmfamily
單元充當中心語
\end{styleqwerty}
\end{listWWviiiNumxivleveli}

\textrm{含單一論元的句子[FF0C?]萃取其中唯一的論元[FF0C?]不含斜格成分[FF0C?]充當中心語。}

\textrm{走空襲个人(躲警報的人)}、\textrm{走新聞个人(跑新聞的人)}、\textrm{跳童个人(跳童乩的人)}、\textrm{走三點半个人(跑三點半的人)}、\textrm{走路个人(逃亡的人)}

\begin{listWWviiiNumxivleveli}
\item \begin{styleqwerty}\rmfamily
單純的修飾關係
\end{styleqwerty}
\end{listWWviiiNumxivleveli}

\textrm{這類的派生詞和謂語無關。只對單一對象加以修飾。}

\textrm{歹手爪个人(手腳不乾淨的人)}、\textrm{歹記持个人(記性差的人)}、\textrm{凍霜个人(小氣的人)}、\textrm{破相个人(殘缺的人)}、\textrm{躼骹个人(高挑的人)}

\begin{listWWviiiNumxivleveli}
\item \begin{styleqwerty}\rmfamily
修飾語和被修飾語為同位關係
\end{styleqwerty}
\end{listWWviiiNumxivleveli}

\textrm{分割厝地个代誌(分割房地的事情)}、\textrm{此幚趁着大錢个緣故(這次賺到大錢的緣故)}

\rmfamily
(本節參閱連 2008)

\subsection{\rmfamily 2.10 重疊式}
\subsubsection{\rmfamily 2.10.1 兩音節的重疊式}

\textrm{在未開始討論重疊式先簡介一下台語的聲調的調類和調值。}

\rmfamily
台語聲調系統和調值圖示如下:

獨用調

\tablefirsthead{}

\tabletail{}
\tablelasttail{}
\begin{tabularx}{\textwidth}{XXXXX} & 平 & 上 & 去 & 入\\
\lsptoprule
上 & 55 (高平) & 51 (高降) & 21 (低降) & 2 (低)\\
下 & 24 (升調) &  & 33 (中平) & 4 (高)\\
\lspbottomrule
\end{tabularx}
\textrm{阿拉伯數字代表調類[FF0C?]上平(1)}、\textrm{上上(2)}、\textrm{上去(3)}、\textrm{上入(4)}、\textrm{下平(5)}、\textrm{下去(7)}、\textrm{下入(8)}[FF0C?]\textrm{括弧中的數字表示調類。表中每個聲調都有調值[FF0C?]以阿拉伯數字表示。數目的多寡反映聲調的高低。每個調類都有獨用調(isolation tone)(如上表)和連用調 (combination tone[FF0C?]或稱sandhi tone)(詳下表)的體現。音節單獨出現(即後頭是停頓)或後接輕聲音節時。念獨用調[FF0C?]後面接另一個非輕音節念連用調。如「真好」tsin}\textrm{\textsuperscript{1}} \textrm{ho}\textrm{\textsuperscript{2}}、\textrm{「好个」ho}\textrm{\textsuperscript{2}} \textrm{e}\textrm{\textsuperscript{0}}\textrm{的「好」念獨用高降調[FF0C?]「好天」的ho}\textrm{\textsuperscript{2}} \textrm{thinn}\textrm{\textsuperscript{1}}\textrm{「好」念獨用高平調。}

連用調

\tablefirsthead{}

\tabletail{}
\tablelasttail{}
\begin{tabularx}{\textwidth}{XXXXX} & 平 & 上 & 去 & 入\\
\lsptoprule
上 & 33 (中平) & 55 (高平) & 51(高降) & 4(高)\\
下 & 33 (中平)或11 (低平) &  & 11 (低平) & 2(低)\\
\lspbottomrule
\end{tabularx}
\textrm{下平和下去調海口腔合流為低平調[FF0C?]通行腔有別。如「橋頭」kio}\textrm{\textsuperscript{5}} \textrm{thau}\textrm{\textsuperscript{5}}\textrm{、「轎頭」kio}\textrm{\textsuperscript{7}} \textrm{thau}\textrm{\textsuperscript{5}}\textrm{。海口腔不論是「橋頭」還是「轎頭」[FF0C?]第一音節都念低平調[FF0C?]分不清到底是「橋」還是「轎」。但通行腔前者念中平調[FF0C?]後者念低平調。}

\textrm{形容詞都有修飾的對象[FF0C?]將對象的某方面的特性表露出來。附在語根後面的重疊型態[FF0C?]主要的功效在形容詞所表示的語意基礎上再更表達與感知相關的生動樣貌[FF0C?]有擬態、擬聲的效應。重疊格式為形容詞+XX (兩音節的重疊)}。\textrm{這類重疊式的語義相當豐富[FF0C?]可參酌母語者加以體會[FF0C?]這裡限於篇幅[FF0C?]不予贅述。}

表2-7 兩音節重疊式

\tablefirsthead{}

\tabletail{}
\tablelasttail{}
\begin{tabularx}{\textwidth}{XXX}
\lsptoprule

 形容詞 & 重疊式 & 羅馬拼音\\
 肥 & 肥至至 & {\sffamily \textrm{pui}\textrm{\textsuperscript{5}} \textrm{tsi}\textrm{\textsuperscript{3}} \textrm{tsi}\textrm{\textsuperscript{3}}}\\
 恬 & 恬咒咒 & {\sffamily \textrm{tiam}\textrm{\textsuperscript{5}} \textrm{tsiu}\textrm{\textsuperscript{3}} \textrm{tsiu}\textrm{\textsuperscript{3}}}\\
 金 & 金示示 & {\sffamily \textrm{kim}\textrm{\textsuperscript{1}} \textrm{si}\textrm{\textsuperscript{3}} \textrm{si}\textrm{\textsuperscript{3}}}\\
 白 & 白拋拋 & {\sffamily \textrm{peh}\textrm{\textsuperscript{8}} \textrm{phau}\textrm{\textsuperscript{1}} \textrm{phau}\textrm{\textsuperscript{1}}}\\
& 白濛濛 & {\sffamily \textrm{peh}\textrm{\textsuperscript{8}} \textrm{bong}\textrm{\textsuperscript{5}} \textrm{bong}\textrm{\textsuperscript{5}}}\\
\hhline{~--} & 白蒼蒼 & {\sffamily \textrm{peh}\textrm{\textsuperscript{8}} \textrm{tshang}\textrm{\textsuperscript{1}} \textrm{tshang}\textrm{\textsuperscript{1}}}\\
 紅 & 紅貢貢 & {\sffamily \textrm{ang}\textrm{\textsuperscript{5}} \textrm{kong}\textrm{\textsuperscript{3}} \textrm{kong}\textrm{\textsuperscript{3}}}\\
& 紅嘰嘰 & {\sffamily \textrm{ang}\textrm{\textsuperscript{5}} \textrm{ki}\textrm{\textsuperscript{3}} \textrm{ki}\textrm{\textsuperscript{3}}}\\
\hhline{~--} & 紅趴趴 & {\sffamily \textrm{ang}\textrm{\textsuperscript{5}} \textrm{pha}\textrm{\textsuperscript{3}} \textrm{pha}\textrm{\textsuperscript{3}}}\\
 䆀 & 䆀猴猴 & {\sffamily \textrm{bai}\textrm{\textsuperscript{2}} \textrm{kau}\textrm{\textsuperscript{5}} \textrm{kau}\textrm{\textsuperscript{5}} \textrm{(醜八怪)}}\\
 重 & 重塊塊 & {\sffamily \textrm{tang}\textrm{\textsuperscript{7}} \textrm{khuai}\textrm{\textsuperscript{5}} \textrm{khuai}\textrm{\textsuperscript{5}}}\\
 硬 & 硬著著 & {\sffamily \textrm{ngi}\textrm{\textsuperscript{7}} \textrm{tu}\textrm{\textsuperscript{3}} \textrm{tu}\textrm{\textsuperscript{3}}}\\
 狹 & 狹星星 & {\sffamily \textrm{ueh}\textrm{\textsuperscript{8}} \textrm{tsinn}\textrm{\textsuperscript{1}} \textrm{tsinn}\textrm{\textsuperscript{1}}}\\
 緊 & 緊肆肆 & {\sffamily \textrm{kin}\textrm{\textsuperscript{2}} \textrm{su}\textrm{\textsuperscript{3}} \textrm{su}\textrm{\textsuperscript{3}}}\\
 直 & 直溜溜 & {\sffamily \textrm{tit}\textrm{\textsuperscript{8}} \textrm{liu}\textrm{\textsuperscript{1}} \textrm{liu}\textrm{\textsuperscript{1}}}\\
 暗 & 暗啁啁 & {\sffamily \textrm{am}\textrm{\textsuperscript{3}} \textrm{tsiu}\textrm{\textsuperscript{3}} \textrm{tsiu}\textrm{\textsuperscript{3}}}\\
 倒 & 倒翹翹 & {\sffamily \textrm{to}\textrm{\textsuperscript{3}} \textrm{khiau}\textrm{\textsuperscript{3}} \textrm{khiau}\textrm{\textsuperscript{3}}}\\
 靜 & 靜悄悄 & {\sffamily \textrm{tsing}\textrm{\textsuperscript{7}} \textrm{tsiau}\textrm{\textsuperscript{3}} \textrm{tsiau}\textrm{\textsuperscript{3}}}\\
 薄 & 薄釐釐 & {\sffamily \textrm{poh}\textrm{\textsuperscript{8}} \textrm{li}\textrm{\textsuperscript{5}} \textrm{li}\textrm{\textsuperscript{5}}}\\
 笑 & 笑微微 & {\sffamily \textrm{tshio}\textrm{\textsuperscript{3}} \textrm{bi}\textrm{\textsuperscript{1}} \textrm{bi}\textrm{\textsuperscript{1}}}\\
 兇 & 兇殆殆 & {\sffamily \textrm{hiong}\textrm{\textsuperscript{1}} \textrm{thai}\textrm{\textsuperscript{2}} \textrm{thai}\textrm{\textsuperscript{2}}}\\
\multicolumn{3}{X}{ 疊韻式 (部分重疊)}\\
 薄 & 薄釐絲 & {\sffamily \textrm{poh}\textrm{\textsuperscript{8}} \textrm{li}\textrm{\textsuperscript{5}} \textrm{si}\textrm{\textsuperscript{1}}}\\
 輕 & 輕猛賞 & {\sffamily \textrm{khin}\textrm{\textsuperscript{1}} \textrm{bang}\textrm{\textsuperscript{2}} \textrm{sang}\textrm{\textsuperscript{2}}}\\
 烏 & 烏羅蘇 & {\sffamily \textrm{oo}\textrm{\textsuperscript{1}} \textrm{loo}\textrm{\textsuperscript{7}} \textrm{soo}\textrm{\textsuperscript{1}}}\\
 暗 & 暗敏摸 & {\sffamily \textrm{am}\textrm{\textsuperscript{3} }\textrm{bin}\textrm{\textsuperscript{7}} \textrm{bong}\textrm{\textsuperscript{1}}}\\
\lspbottomrule
\end{tabularx}
\subsubsection{\rmfamily 2.10.2 三音節的重疊}

\textrm{三音節的輸入原聲調產生不同的調值[FF0C?]三音節的連調變化多少有固定的調形[FF0C?]音節長度為「長短中」。三音節的重疊式連用調型與上述的連用調不盡相同。如下表所示[FF0C?]聲律類型與單字調類有某種特定的關係。}

\rmfamily
表2-8 三音節重疊式

\tablefirsthead{}

\tabletail{}
\tablelasttail{}
\begin{tabularx}{\textwidth}{XXXX}
\lsptoprule

{\sffamily \textrm{單字}\textrm{\textbf{調類}}} & 例子 & 羅馬拼音 & 聲律類型\\
 上平調 & 金金金 & {\sffamily \textrm{kim}\textrm{\textsuperscript{1} }\textrm{kim}\textrm{\textsuperscript{1}} \textrm{kim}\textrm{\textsuperscript{1}}} & 5 7 1\\
 下平調 & 圓圓圓 & {\sffamily \textrm{inn}\textrm{\textsuperscript{5}} \textrm{inn}\textrm{\textsuperscript{5}} \textrm{inn}\textrm{\textsuperscript{5}}} & 5 3/7 5\\
 上上調 & 苦苦苦 & {\sffamily \textrm{khoo}\textrm{\textsuperscript{2}} \textrm{khoo}\textrm{\textsuperscript{2}} \textrm{khoo}\textrm{\textsuperscript{2}}} & 1 1 2\\
 上去調 & 細細細 & {\sffamily \textrm{sue}\textrm{\textsuperscript{3}} \textrm{sue}\textrm{\textsuperscript{3}} \textrm{sue}\textrm{\textsuperscript{3}}} & 2 2 3\\
 下去調 & 厚厚厚 & {\sffamily \textrm{kau}\textrm{\textsuperscript{7}} \textrm{kau}\textrm{\textsuperscript{7}} \textrm{kau}\textrm{\textsuperscript{7}}} & 5 3 7\\
 上入調 & 密密密 & {\sffamily \textrm{bat}\textrm{\textsuperscript{4}} \textrm{bat}\textrm{\textsuperscript{4}} \textrm{bat}\textrm{\textsuperscript{4}}} & 5 4 8/4\\
 下入調 & 薄薄薄 & {\sffamily \textrm{poh}\textrm{\textsuperscript{8}} \textrm{poh}\textrm{\textsuperscript{8}} \textrm{poh}\textrm{\textsuperscript{8}}} & 5 4 8/4\\
\lspbottomrule
\end{tabularx}
\subsection{\textrm{2.11 台灣南島語的形態}}
\begin{styleii}
台灣南島語是「膠著」(agglutinative)語\textsf{[FF0C?]}形位基本上不獨用\textsf{[FF0C?]}但形位界線清楚\textsf{[FF0C?]}這點和「孤立」(isolating)語或「屈折」(inflectional)語不相同\textsf{。}孤立語中語詞大多沒有形態變化\textsf{[FF0C?]}而屈折語通常是幾個語法範疇融合於一個形位中\textsf{[FF0C?]}如英語的her將人稱\textsf{、}性\textsf{、}數\textsf{、}格位四個語法範疇融合在一起\textsf{[FF0C?]}不能分離出來成為個別的形式\textsf{。}
\end{styleii}

\begin{styleii}
膠著語以語根為基石[FF0C?]以派生的方式即「加綴」(affixation)\textsf{[FF0C?]}包括加前綴、中綴、後綴來組句[FF0C?]詞綴都是封閉類數量有限的語法範疇[FF0C?]這些語法範疇反映一定的線性順序和階層高低[FF0C?]大體上越靠近語根的內層其階層越低[FF0C?]越遠離語根的外層其階層越高。由內層到外層由低層到高層約略可以理出一套階層出來。內層動相[FF0C?]外層體貌、致使、語態、時制、語勢(含各種句式[FF0C?]如[FF1A?]祈使、疑問、感嘆、虛擬等)、人際互動的言外行為層。這是指當下言談事件中說者和聽者間現場的互動[FF0C?]比如祈使式是說者驅使聽者以說者的意志行事[FF0C?]疑問式(如正反問句)是說者期待聽者就正反兩個選項給予答覆[FF0C?]其他句式也涉及聽說者之間的互動。
\end{styleii}

\textrm{「派生」(derivation)是語根「加詞綴」(affixation)的造詞方式。詞綴包括前綴、後綴、中綴。前兩類漢語不乏例證[FF0C?]但中綴很罕見[FF0C?]南島語中這類的詞綴則相當發達。台灣南島語和漢語都是人類的語言[FF0C?]所表達的概念範疇固然有諸多雷同之處[FF0C?]但表達的方式卻大異其趣。台灣南島語詞法和句法常交織在一起[FF0C?]最引人入勝的是語根內的「中綴」(infix)和句法、語意有密切的互動關係。}

\textrm{台灣南島語反映出形態現象的多樣性[FF0C?]特別是加綴和重疊[FF0C?]詳參第七章7.3.節的論述。}以下茲就形態和語法的互動和指代詞的形態變化兩方面略加論述。

\subsubsection{\textrm{2.11.1 形態和語法的互動}}

\textrm{台灣南島語中以動詞語根為核心向前後展延出詞綴(即前綴、後綴)}[FF0C?]\textrm{表達「動相」(aktionsart)}、\textrm{「體貌」(aspect)}、\textrm{「致使」(causativity)}、\textrm{「情態」(modality)}、\textrm{「語態」(voice)}、\textrm{「語氣」(mood)等各種語法範疇。語根的中綴反映句子「語態」(voice)的選擇及論元間的語法關係[FF0C?]還涉及論元的語意角色[FF0C?]包括兩種主要角色。核心者為施受關係[FF0C?]周邊者為憑藉、方位等斜位關係[FF0C?]比如選取「施事語態」(actor voice)中綴[FF0C?]施事論元須出現於主語的位置[FF0C?]並須帶主格標記[FF0C?]如選取非施事語態中綴(分受事、處所、參考三種)}[FF0C?]\textrm{受事、處所或參考就常出現於主語的位置[FF0C?]也須帶各自的格位標記。除了表現施受的核心及其他周邊的關係[FF0C?]還需表現語氣、情態、致使、體貌關係[FF0C?]包括直陳語氣(有「實存」(realis)與「非實存」(irrelais)之別)}、\textrm{非直陳語氣包括「勸誘式」(hortative)}、\textrm{「祈願式」(optative)}、\textrm{甚至「示證式」(evidential)等[FF0C?]以上簡述的表現法都可以在台灣南島語賽德克語中找到。比如下所示例 (Zeitoun, \citealt{ChuKaybaybaw2015}: 310)}[FF0C?]\textrm{賽夏語si’ael是動詞語根「吃」[FF0C?]要表某人(即「施事」(agent))吃東西[FF0C?]需在語根中適當的位置(即緊跟在第一個音段之後)插入表「施事語態」(AV[FF0C?]即actor voice)的中綴om}[FF0C?]\textrm{主語須帶「主格」(Nom)標記。賽夏語代名詞是一個「活用」(inflectional)系統[FF0C?]其形態隨語法範疇(人稱、單複數、格位、語氣(包括否定)而變[FF0C?]以「融合的」(fusional)形式表現。主格第一人稱單數為yako[FF0C?]第三人稱單數為sia[FF0C?]兩者看不出形態的關係。}

\begin{listWWviiiNumviiileveli}
\item \begin{styleqwerty}\rmfamily
Yako  s <om>i’ael  soe’hae’  ka  walo’
\end{styleqwerty}
\end{listWWviiiNumviiileveli}
\rmfamily
  第一人稱單數主格  <施事語態>吃  一  鏈接  糖果

 \textrm{“我吃一個糖果}\textrm{\textbf{”}}

\textrm{南島語(賽夏語除外)句中的語序絕大部分都是主動詞居於句首[FF0C?]其次為動詞、再次為名詞和其他成分。動詞帶各種焦點標記標出句中扮演某種語義角色(如施事、受事、憑藉、方位等)作為焦點[FF0C?]動詞後成為焦點的名詞需標上格位標記[FF0C?]分為主格(或可稱為正格)與前方動詞的焦點標記相呼應[FF0C?]其他名詞標上斜格(或可稱為偏格)}[FF0C?]\textrm{有些南島語還從斜格分出屬格[FF0C?]以標明不構成焦點的施事和其他非施事的斜格有所區隔。}

\textrm{如暫時撇開其他功能範疇不論[FF0C?]如果動詞帶兩個「論元」(argument)時[FF0C?]而這兩個論元表示施事、受事關係[FF0C?]句首動詞中的活用形態標示施事為焦點時[FF0C?]動後的施事名詞標為正格[FF0C?]跟它呼應[FF0C?]受事標為斜格。要是活用形態標示受事為焦點時[FF0C?]受事名詞會帶上正格、施事帶上屬格。動詞帶三個論元時[FF0C?]即傳統所謂的雙賓動詞[FF0C?]要是動詞中的活用形將間接賓語(即接受者)標示為處所(或可稱為目標)焦點[FF0C?]處所(即接受者)就相應的標成正格[FF0C?]給予者標為屬格[FF0C?]而客體(即直接賓語)標成為斜格。以下就這三種情況各舉一例(黃、施 2008):}

\rmfamily
焦點結構和格位標記(布農語)

\rmfamily
施事為焦點

\begin{listWWviiiNumviiileveli}
\item \begin{styleqwerty}\rmfamily
ma’anat  a  tina=in  a  bainu
\end{styleqwerty}
\end{listWWviiiNumviiileveli}

 \textrm{施事焦點-煮  主格  媽媽=限定詞  主格  豆子}

 \textrm{媽媽煮豆子 (主事焦點主語[FF1A?]施事)}

\rmfamily
受事為焦點

\begin{listWWviiiNumviiileveli}
\item \begin{styleqwerty}\rmfamily
na=’anat-un=ku  a  bainu=a
\end{styleqwerty}
\end{listWWviiiNumviiileveli}

 \textrm{非實現=煮-受事焦點=我.屬格  主格  豆子=限定詞.主格}

 \textrm{那些豆子我要煮 (受事焦點主語[FF1A?]受事)}

\rmfamily
處所為焦點

\begin{listWWviiiNumviiileveli}
\item \begin{styleqwerty}\rmfamily
na=saiv-an=ku  a  subali=a  mas  bainu
\end{styleqwerty}
\end{listWWviiiNumviiileveli}

 \textrm{非實現=給-處所焦點{}-我.屬格  主格  人名=限定詞.主格  斜格  豆子}

 \textrm{我要給Subali豆子 (處所焦點句主語[FF1A?]接受者)}

\textrm{從上述的解析可以看出南島語句子三維合一的豐富面貌[FF01?]一個句子同時涵蓋焦點結構、各種語意角色、語法關係(主斜或正偏格位關係)}。\textrm{在上述三合一的綜合體還需加上語氣(分實存和非實存兩式)才算完整[FF0C?]此外句子還涉及體貌系統的表現。以上可以看出形態和語意、句法、信息結構的相互作用。}

\subsubsection{\textrm{2.11.2 指代詞的形態變化}}

\textrm{指代詞包括指示代詞和人稱代詞[FF0C?]兩者都是語言系統中的封閉類語詞[FF0C?]可以窮舉[FF0C?]屬功能範疇。}

\textrm{南島語人稱代詞有其特色[FF0C?]試舉布農語略加說明。人稱代詞反映語言系統中「活用」(inflectional)的特徵[FF1A?]活用即語法範疇的融合[FF0C?]語法範疇及其屬性羅列如下:人稱(第一、二、三人稱)、數(單、複數)}、\textrm{格(主格、所有格、斜格/屬格、處所格)}。\textrm{第一人稱分包括、排除兩式。代詞形態分自由、黏著兩式[FF0C?]自由式可以脫離動詞獨用[FF0C?]而黏著式非附著於動詞不可。第一、二人稱共現時有三種融合式。形態的體現是在每個語法範疇中選取相關的屬性並匯集起加以融合而成。其運作方式可由下表體會出來。}

\textrm{表2-10 人稱}\textrm{\textbf{代名詞} (}高雄郡社布農語)

\tablefirsthead{}

\tabletail{}
\tablelasttail{}
\begin{tabularx}{\textwidth}{XXXXXXXXX}
\lsptoprule

\multicolumn{3}{X}{ 人稱代名詞} & \multicolumn{4}{X}{ 自由式} & \multicolumn{2}{X}{ 黏著式}\\
\multicolumn{1}{X}{ 數} & \multicolumn{2}{X}{ 人稱} & 主格 & 所有格 & 斜格/屬格 & 處所格 & 主格 & 斜格/屬格\\
\multicolumn{1}{X}{單數} & \multicolumn{2}{X}{一} & saikin & inaak & (ma)zaku & zakuan & ik & ku\\
& \multicolumn{2}{X}{二} & kasu(n) & isuu & (ma)suu & suuan & as & su\\
\hhline{~--------} & \multicolumn{2}{X}{三} & saia

sia(n) & isaicia

isia & (ma)saicia & si-aan cia & {}- & {}-\\
\multicolumn{1}{X}{複數} & \multicolumn{1}{X}{一} & 包含式 & kata & imita & (ma)ita

mita & mitaan & ta & ta\\
&  & 排除式 & kaimin & inaam & (ma)zami & zamian & im & {}-\\
\hhline{~--------} & \multicolumn{2}{X}{二} & kamu(n) & imuu & (ma)muu & muuan & am & mu\\
\hhline{~--------} & \multicolumn{2}{X}{三} & naia/nai/nian & inaicia/inai & (ma)naicia/(ma)nai & naian cia & {}- & {}-\\
\hhline{~--------}
\lspbottomrule
\end{tabularx}
表2-11 人稱代名詞 (賽德克語)

\tablefirsthead{}

\tabletail{}
\tablelasttail{}
\begin{tabularx}{\textwidth}{XXXXXXXX}
\lsptoprule

\multicolumn{3}{X}{ 人稱代名詞} & \multicolumn{2}{X}{ 黏著式} & \multicolumn{3}{X}{ 自由式}\\
\multicolumn{1}{X}{ 數} & \multicolumn{2}{X}{ 人稱} & 主格 & 屬格 & 中性格 & 所有格 & 斜格\\
\multicolumn{1}{X}{單數} & \multicolumn{2}{X}{一} & ku & mu & yaku & nnaku/naku & kenan/muna\\
& \multicolumn{2}{X}{二} & su & su & isa & nnisu/nisu & sunan\\
\hhline{~-------} & \multicolumn{2}{X}{三} & {}- & na & heya & nheya & {}-\\
\multicolumn{1}{X}{複數} & \multicolumn{1}{X}{一} & 包含式 & ta & ta & ita & nnita/nita & {}-\\
&  & 排除式 & nami-miyan & nami-miyan & yami & nnami & {}-\\
\hhline{~-------} & \multicolumn{2}{X}{二} & namu & namu & yamu & nnamu & {}-\\
\hhline{~-------} & \multicolumn{2}{X}{三} & {}- & daha & dheya & ndheya & {}-\\
\multicolumn{3}{X}{融合式 (portmanteau)} & \multicolumn{2}{X}{{\sffamily \textrm{misu 我}\textrm{\textsubscript{主事者}}\textrm{+你}\textrm{\textsubscript{受事者}}}

{\sffamily \textrm{saku 你}\textrm{\textsubscript{主事者}}\textrm{+我}\textrm{\textsubscript{受事者}}}

{\sffamily \textrm{maku 我}\textrm{\textsubscript{主事者}}\textrm{+你們}\textrm{\textsubscript{受事者}}}} & {}- & {}- & {}-\\
\lspbottomrule
\end{tabularx}
\textrm{指示代詞含蓋數量較少的語法範疇[FF0C?]包括以說者座落為基準的遠指和近指的二分法(有些南島語如布農語分得更細[FF0C?]有遠、中、近指的三分法)[FF0C?]另外依觀察者為角度分成可見和看不到兩類。遠近指和可見/見不到兩個交錯成指示詞的形態系統。詳下表:}

表2-12 指示代詞 (賽夏語)

\tablefirsthead{}

\tabletail{}
\tablelasttail{}
\begin{tabularx}{\textwidth}{XXXXXXX}
\lsptoprule

\multicolumn{4}{X}{近 < -{}-{}-{}-{}-{}-{}-{}-{}-{}-{}-{}-{}-{}-{}-{}-{}-{}-{}-{}-{}-{}-{}-{}-{}-{}-{}-{}-{}-{}-{}-{}-{}-{}-{}-{}- >遠} & \multicolumn{3}{X}{近 < -{}-{}-{}-{}-{}-{}-{}-{}-{}-{}-{}-{}-{}-{}-{}-{}-{}-{}-{}-{}-{}-{}- >遠}\\
hini & hiSon & hiza & hita & hani & haSon & hato\\
這裡 & 這邊 & 那邊 & 那邊 & 這附近 & 那邊 & 那邊\\
可見 & 可見 & 可見 & 可見 & 看不到 & 看不到 & 看不到\\
\multicolumn{4}{X}{hi-可見} & \multicolumn{3}{X}{ha-看不到}\\
\lspbottomrule
\end{tabularx}
\subsection{\rmfamily 2.12 摘要與結論}
\subsubsection{\rmfamily 2.12.1 摘要}

\textrm{本文探討形態學(或稱「詞法」)的各個面向[FF0C?]含蘊的主題包含語詞和詞組的區別、音位和形態音位、形態和語言系統其他部門(如詞匯、句法、語用)的互動、複合和派生的區別、從複合到派生的演變、形態和語言層次互動、「个」的派生功能、重疊式。本文主要以台語為素材[FF0C?]探索形態學的各個面相[FF0C?]並簡略的介紹台灣南島語的若干形態特性並梳理出形態和語言部門的交互作用[FF0C?]這只觸及冰山的一角[FF0C?]其豐富的內涵有待讀者來發掘。}

\subsubsection{\rmfamily 2.12.2 結論}

\textrm{本文著重台灣閩南語的形態學探索[FF0C?]並對台灣南島語的形態稍加徵引。形態運作是語言系統中不可或缺的一環[FF0C?]形態與語言系統中的其他部門有密切的互動關係[FF0C?]舉凡詞匯、音韻、句法、語用等都牽涉在其中。形態的生成有幾種方式[FF0C?]如複合、派生、重疊[FF0C?]這都是台語和中文常見的作業方式[FF0C?]唯獨沒有屈折(或稱活用)的運作方式。台灣南島語的形態有特異性[FF0C?]特別是和語義、語法、信息結構有密不可分的交互作用。此外[FF0C?]我們可從歷時演變的角度探討複合如何變成派生[FF0C?]並和語義的延伸息息相關。}

\subsection{\rmfamily 2.13 參考文獻}

\textrm{趙元任(2002)[FF0C?]<中國話的文法 (增訂版)>[FF0C?]丁邦新譯。香港[FF1A?]香港中文大學出版社。}

\textrm{李訥、安珊笛(2008)[FF0C?]<漢語語法 (Mandarin Chinese: A Functional Reference Grammar)[FF08?]修訂版[FF09?]>[FF0C?]黃宣範(譯)[FF0C?]台北[FF1A?]文鶴出版有限公司。}

\textrm{連金發(1998)[FF0C?]<台灣閩南語詞綴‘仔’的研究>[FF0C?]黃宣範(編)[FF0C?]<<第二屆台灣語言國際研討會論文選集>}\textrm{>}\textrm{[FF0C?]465-483。台北[FF1A?]文鶴出版有限公司。}

\textrm{連金發(1999)[FF0C?]<台灣閩南語<頭>的構詞方式>[FF0C?]殷允美、楊懿麗、詹惠珍(編)[FF0C?]<<中國境內語言暨語言學{\textbullet}第五輯{\textbullet}語言中的互動>>[FF0C?]289-309[FF0C?]中央研究院語言學研究所籌備處會議論文輯之二。}

\rmfamily
連金發(2000)[FF0C?]<構詞學問題探索>[FF0C?]<<漢學研究>>[FF0C?]18[FF0C?]61-78。

\textrm{連金發(2008)[FF0C?]<台灣閩南語的‘个’研究>[FF0C?]張郇慧、連金發、蘇以文 (合編)[FF0C?]<<一步一腳印鄭良偉教授榮退論文集>>[FF0C?]143-170。台北[FF1A?]文鶴出版有限公司。}

\textrm{黃慧娟、施朝凱(2018)[FF0C?]<布農語語法概論>[FF0C?]新北市[FF1A?]原住民族委員會。}

\textrm{宋麗梅(2018)[FF0C?]<賽德克語語法概論>[FF0C?]新北市[FF1A?]原住民族委員會。}

\textrm{葉美利(2018)[FF0C?]<賽夏語語法概論>[FF0C?]新北市[FF1A?]原住民族委員會。}

\textrm{Aronoff, Mark, and Kirsten Fudeman. 2011.} \textrm{\textbf{\textit{What} \textit{is} \textit{morphology?}}} \textrm{Malden: Wiley-Blackwell.}

\textrm{Bauer, Laurie. 1983.} \textrm{\textbf{\textit{English} \textit{word-formation}}}\textrm{. Cambridge: Cambridge University Press.}

\textrm{Bauer, Laurie. 2004.} \textrm{\textbf{\textit{A} \textit{glossary} \textit{of} \textit{morphology}}}\textrm{. }\textrm{Washington D.C: Georgetown University Press.}

\textrm{Bloomfield, Leonard. 1933.}\textrm{\textbf{ \textbf{Language}}}\textrm{.} \textrm{New York: Henry Holt.}

\textrm{Chao, Yuan Ren. 1968.} \textrm{\textbf{\textit{A} \textit{grammar} \textit{of} \textit{spoken} \textit{Chinese}}}\textrm{. Berkeley: University of California Press.} 

\textrm{Chung, Karen Steffen.2006.}\textrm{ }\textrm{\textbf{\textit{Mandarin} \textit{compound} \textit{verbs}}}\textrm{. Taiwan Journal of Linguistics Book Series in Chinese Linguistics. Taipei: Crane Publishing Company}

\textrm{Harley, Heidi. 2006.}\textrm{ }\textrm{\textbf{\textit{English} \textit{words:} \textit{A} \textit{linguistic} \textit{introduction}}}\textrm{.} \textrm{Malden: Blackwell Publishing.}

\textrm{Hippisley, Andrew, and Gregory Stump (eds.). 2016.} \textrm{\textbf{\textit{The} \textit{Cambridge} \textit{handbook} \textit{of} \textit{morphology}}}\textrm{. Cambridge: Cambridge University Press.}

\textrm{Huang, Shuanfan. 1998. Chinese as a headless language in compounding morphology.} \textrm{\textbf{\textit{New} \textit{approaches} \textit{to} \textit{Chinese} \textit{word} \textit{formation:} \textit{Morphology,} \textit{phonology} \textit{and} \textit{the} \textit{lexicon} \textit{in} \textit{ancient} \textit{and} \textit{modern} \textit{Chinese}}}\textrm{, }\textrm{ed. by Jerome L. Packard, 261-283. Berlin \& New York: Mouton de Gruyter.}

\textrm{Li, Charles N., and Sandra A. Thompson. 1989.} \textrm{\textbf{\textit{Mandarin} \textit{grammar.} \textit{A} \textit{functional} \textit{reference} \textit{grammar}}}\textrm{.} \textrm{Berkeley: University of California Press.}

\textrm{Lien, Chinfa. 2001. Competing morphological changes in Taiwan Southern Min.} \textrm{\textbf{\textit{Sinitic} \textit{grammar:} \textit{Synchronic} \textit{and} \textit{diachronic} \textit{perspectives}}}\textrm{,} \textrm{ed. By Hilary Chappell. 309-339. London: Oxford University Press.}

\textrm{Matthews, P. H. 1991.} \textrm{\textbf{\textit{Morphology}}}\textrm{. Cambridge: Cambridge University Press.}

\textrm{Norman, Jerry, and Tsu-lin Mei. 1976. The Austroasiatics in ancient south China: Some lexical evidence.} \textrm{\textbf{\textit{Monumenta} \textit{Serica}}} \textrm{32.274-301.}

\textrm{O’Neill, Paul. 2013. Morphomes and morphosyntactic/semantic features.} \textrm{\textbf{\textit{Boundaries} \textit{of} \textit{pure} \textit{morphology}}}\textrm{, ed. by Silvio Cruschina; Martin Maiden; and John Charles Smith, 221-247. Oxford: Oxford University Press.}

\textrm{Packard, Jerome L. 2000.} \textrm{\textbf{\textit{The} \textit{morphology} \textit{of} \textit{Chinese}}}\textrm{.} \textrm{Cambridge: Cambridge University Press.}

\textrm{Stump, Gregory T. 2001.}\textrm{ }\textrm{\textbf{\textit{Inflectional} \textit{morphology:} \textit{A} \textit{theory} \textit{of} \textit{paradigm} \textit{structure}}}\textrm{. }\textrm{Cambridge studies in linguistics. Cambrdige: Cambridge University Press.}

\textrm{Zeitoun, Elizabeth; Tai-hwa Chu; and Lalo a tahesh kaybaybaw. 2015.} \textrm{\textbf{\textit{A} \textit{study} \textit{of} \textit{Saisiyat} \textit{morphology}}}\textrm{. Honolulu: University of Hawaiʻi Press.}


\begin{verbatim}%%move bib entries to  localbibliography.bib
\end{verbatim} 