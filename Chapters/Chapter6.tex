\begin{document}

\chapter{Conclusions}
\textcite{renouf2002time} recognizes the importance of digitally storing both historical and modern textual data. ``We need the past in order to understand the present. An amalgamation would increase the scope, timespan and continuity of resources, whilst lessening the inconvenience of having to switch from one corpus and set of tools to another'' \parencite{renouf2002time}. Although automatic data crawling and real-time processing makes it possible to build a dynamic, chronological web corpus of present-day language use, as written texts comprise a major part of existing corpora, it is a turning point to explore the diachrony of the data along with the lately available texts from historical periods.

In light of the growing interest in diachronic lexical semantic change, this paper is a case-study investigation of \jia through a corpus-based approach. is Language does not cease to change beyond the observable texts within the time frame of the chosen corpora, and  

The evolution of jia is a compressed history of the Chinese society and the Chinese language. The analysis of word representations of jia serves as a starting point to pinpoint the core, stable meanings of the word, outlining the properties of a physical space and a structured social unit. While the emphasis has been put on the economic situation from pre-modern time, the word jia becomes less associated with individuated roles such as a wife, but more closely focused on the self, depicting personal memories of home leaving and returning. 

With the advantage of distributional semantic models, the meaning conflation of home, house, and family can be explored as different components. Especially, premodern Chinese is distinguished from the current written form, uses different lexical items, and is mostly in the form of one syllable. The disparity results in the addition of new senses of the one-character jia, and aspects of meanings are encoded in different two-character words in modern time. In the field of corpus and computational linguistics, changes of word choice and the inclusion of more senses allow for a closer look at the texts in snapshots of specific time frames, while resonates with studies in other disciplines.

How polysemy of homophone is to be explored through external resources such as dictionary and negative examples \textcite[15]{traugott2001regularity}.  Cross-linguistic and metalinguistic analyses are insightful. In addition, as change in meaning is ongoing, the detection of semantic change can be detected in progress.
\end{document}