\begin{document}

\chapter{Conclusions}
In light of the growing interest in diachronic lexical semantic change, this paper is a case-study investigation of \jia through a corpus-based approach. is Language does not cease to change beyond the observable texts within the time frame of the chosen corpora, and  

The evolution of jia is a compressed history of the Chinese society and the Chinese language. The analysis of word representations of jia serves as a starting point to pinpoint the core, stable meanings of the word, outlining the properties of a physical space and a structured social unit. While the emphasis has been put on the economic situation from pre-modern time, the word jia becomes less associated with individuated roles such as a wife, but more closely focused on the self, depicting personal memories of home leaving and returning. 

With the advantage of distributional semantic models, the meaning conflation of home, house, and family can be explored as different components. Especially, premodern Chinese is distinguished from the current written form, uses different lexical items, and is mostly in the form of one syllable. The disparity results in the addition of new senses of the one-character jia, and aspects of meanings are encoded in different two-character words in modern time. In the field of corpus and computational linguistics, changes of word choice and the inclusion of more senses allow for a closer look at the texts in snapshots of specific time frames, while resonates with studies in other disciplines.

How polysemy of homophone is to be explored through external resources such as dictionary and negative examples \textcite[15]{traugott2001regularity}.  Cross-linguistic and metalinguistic analyses are insightful. In addition, as change in meaning is ongoing, the detection of semantic change can be detected in progress.

As discussed in \textcite{giulianelli2019lexical}, the fine-tuning of large-scaled pre-trained language models like BERT does not yield satisfactory results of temporal-specific contextualized usage/token representations. As hinted by \textcite{giulianelli2019lexical}, the fine-tuning is based on classification task of reconizing the time period of a portion of documents, but the fine-tuned models might instead reflect the style of prominent authors of certain time periods, reering away from baseline representations. Faced with these problems, \textcite{kutuzov2020uio} also compares contextualized embeddings with context-independent ones, and find that for semantic change detection, context-independent embeddings are effective.

Semantic change modeling has profound impacts in linguistic analysis. As language is a dynamic phenomenon, a temporal-aware understanding is explored as a starting point. Following the examination of factors, sense evolution prediction, the interaction between semantic change and different linguistic, cultural factors can deepen our understanding.

However, the character-based embeddings serve as a starting point to investigate the semantic development of Chinese, which is so distinctively different in pre-modern and modern time that calls for an integration of the disyllabic development of Chinese to account for the differences in different time periods. Recently, dependency parser of pre-modern Chinese has been released, yet the segmentation still split many disyllabic words into units of single characters. Nonetheless, through the analysis of different measures of semantic change, this study captures different aspects of semantic properties, and it is hoped that the results can lay an empirical basis of how single characters behave semantically by considering the time dimension of the textual data. In conclusion, this study aims to explore the word representations that are more dynamic than present application is populated for, and to show how word co-occurences can be revealing in terms of such a concept like home that is relatively stable but ever-evolving with the passage of time.
\end{document}