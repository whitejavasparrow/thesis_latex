\begin{document}

\chapter{Introduction}
Language is constantly changing and evolving. The emergence of new senses, the demise of old ones, and the polysemous nature of linguistic expressions make the process of semantic change a dynamic phenomenon \parencite{robertinvanhove2008}. As individuals learn new words and meanings throughout their life, so does a language. As language users actively engage in processing and interpreting the language, the semantic history of words are woven into the texts that then survive time and are presented to us now. In the long run, a word is likely to convey a meaning completely different or unfathomable. For instance, ``the quick and the dead'', quoted from the Bible, means ``the living and the dead'', but the collective adjective ``the quick'' no longer makes sense in Present-Day English \parencite[199]{semanticincrowley2010}.

\textcite{renouf2002time} reflects on how textual data starts to be treated more than ``a static entity.'' In \cite*{sinclair1982reflections}, \citeauthor{sinclair1982reflections} envisions the possibility of ``vast, slowing changing stores of text'' and ``detailed evidence of language evolution'' \asincite{renouf2002time}. The use of digitalized libraries as rich linguistic resources to observe how certain linguistic features are ``assimilated'' into the language becomes more and more feasible \parencite{renouf2002time}. While recent studies have used time-sliced collections of texts to observe swift meaning changes, the digitalization of texts from earlier time periods opens up research opportunities that incorporates a corpus-driven approach to trace the diachronic development of words and their meanings \parencite{kutuzov2018survey,tahmasebi2018survey,camacho2018survey}.

Additionally, the change in meaning is captured by translating discrete linguistic data into numeric vectors such as word embeddings, especially after the release of Word2vec \parencite{mikolov2013efficient}, GloVe \parencite{pennington2014glove} and FastText \parencite{bojanowski2016enriching}. An initial attempt is to generate word embeddings from different time spans and explore whether semantic change occurs based on the neighboring words of the target word from each time period.

The concept of home is an ancient, seemingly familiar and encompassing, but tangible one. Various humanities disciplines have sought to grasp the full picture. Defined by the Oxford English Dictionary (OED), the word \textit{home} is ``the place where a person or animal dwells'' \dictcite{homeinoed}. As one of the earliest 1\% entries to be included in the OED, this word has 35 main senses and 214 total senses—Home is a physical space, a place where we feel a ``sense of belonging [and] comfort'', and even a person's ``country or native land.'' In Mandarin Chinese, the MOE Revised Mandarin Chinese Dictionary defines its translated equivalent \zh{家}{jiā}{home} as ``a place where family members live together (眷屬共同生活的場所)'', ``a private property (私有財產)'', and ``people in certain professional fields (經營某種行業或具有某種身份的人)'' \dictcite{jiainmoe}. Yet, how is the concept of home encoded linguistically? Specifically, how is diachrony interacts with synchrony and variations?

From the perspective of corpus-based computational linguistics, questions are invoked as to how the concept of home is understood by the computer? What words are semantically related to this concept? Because we live at home, far from home, or in a place we call home, and we are constantly searching for the meanings of home, this study aims to explore how semantically-related words construct the meanings of home, and how this concept comes into shape through the lens of time.

This paper is organized as follows. An overview and reflections of semantic change and diachronic word embeddings are given in section~\ref{related_works}. The development of word-level and sense-level word representations brings to the fine-grained analyses and generalizations of semantic change. The topic of home is introduced in the second section. Section 3 describes how semantic change is captured and visualized.

\end{document}