\begin{document}

\chapter{Discussion}
Following \textcite{hamilton2016law}, in which the evaluation is based on examples from previous works on semantic change and words with the ``obsolete'' tag in the Oxford English Dictionary (OED), dictionary entries are consulted to look for ``舊時'' and ``古代'' for attested examples to evaluate the trained diachronic word embeddings.

For example, \zh{齒}{chǐ}{tooth} used to carry the meaning `age (年齡)' and `being of equal rank (並列)' because age determination is made by numbering horses' teeth, which emerges one each year, as in `子之齒長矣,不能事人 (You are long in the tooth)' and `不敢與諸任齒 (I would not dare to take rank equivalent to yours)'; another example is \zh{卑鄙}{bēi-bǐ}{despicable}, which is more neural in connotation in the past \parencite[前言]{王1997古}.

The meanings are based on 漢語大字典, 漢語大詞典, 辭源, 辭海 as well as 現代漢語詞典 and 新華詞典 (both published by 商務印書館).

frequency data is derived from 在线古代汉语语料库字频数据\footnote{\url{http://corpus.zhonghuayuwen.org/resources.aspx}} and 近代漢語語料庫詞頻統計\footnote{\url{https://elearning.ling.sinica.edu.tw/jindai.html}}

\end{document}